\section{Introducción}

El transporte público urbano a menudo presenta una problemática por una mala experiencia del usuario principalmente marcada por la falta de información sobre las rutas y ubicación de los vehículos de transporte, lo que resulta en falta de interés y de información precisa.

UrbanTracker nace para cerrar esta problemática sobre la desinformación, permitiendo tanto a los administradores como al público tener acceso a la ubicación e información precisa sobre los vehículos de transporte. Un Sistema de Geolocalización de Rutas de Transporte, teniendo como principal desafío técnico cómo seleccionar un protocolo de comunicación que pueda manejar un volumen elevado de mensajes (coordenadas GPS), de manera continua y cómo obtener la ubicación de los vehículos con un bajo costo y consumo. Se abordó esta problemática ofreciendo una solución accesible que utiliza los dispositivos móviles de los conductores como GPS para enviar datos de geolocalización. Para mostrar una visualización en tiempo real para los usuarios y una actualización constante de sus posiciones. Se requiere la integración de algunas tecnologías IoT (Internet de las Cosas) para la comunicación eficiente e ágil para una baja latencia, utilizando el protocolo MQTT para la comunicación.

El principal propósito del presente documento es validar e informar por qué se ha hecho la selección del protocolo MQTT (protocolo de envío ligero y eficiente) para la eficiente comunicación en tiempo real en la arquitectura de UrbanTracker, basándose en el exhaustivo análisis de investigación de trabajos previos en el campo de las tecnologías IoT (Internet de las Cosas) y la geolocalización.

En un contexto urbano moderno, donde la movilidad sostenible es crucial, sistemas como UrbanTracker no solo mejoran la accesibilidad al transporte público, sino que también facilitan la planificación urbana inteligente, reduciendo emisiones y optimizando el flujo vehicular.