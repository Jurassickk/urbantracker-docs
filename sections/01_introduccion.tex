\section{Introducción}
El transporte público urbano suele enfrentarse a problemas de ineficiencia operativa y falta de información para los usuarios. En muchos sistemas actuales los pasajeros desconocen en tiempo real la ubicación de los autobuses, lo que genera esperas impredecibles. Sistemas de seguimiento de autobuses en tiempo real pueden reducir estos tiempos de espera y mejorar la planificación de viajes. Además, investigaciones sobre aplicaciones de rastreo con datos precisos han mostrado que la información exacta de llegada de vehículos aumenta la confianza de los usuarios y fomenta el uso de transporte sostenible.

En este contexto, UrbanTracker se plantea como una solución integral de geolocalización de flota urbana. Su objetivo es permitir que los usuarios consulten rutas disponibles y vean en un mapa la posición en tiempo real de los vehículos de transporte en servicio. Para ello, el sistema integra con servicios de mapas y tecnologías GPS, y emplea protocolos de comunicación en tiempo real (WebSockets o MQTT) para la transmisión continua de datos de ubicación. De acuerdo con la especificación de requisitos, el sistema ofrece funcionalidades clave: consulta de rutas, visualización de vehículos activos en mapa, autentificación de conductores para iniciar rutas, y un módulo de administración de rutas, conductores y vehículos. A continuación, se detallan el fundamento teórico, el enfoque de desarrollo, los resultados obtenidos y las conclusiones del proyecto UrbanTracker.