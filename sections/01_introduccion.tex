\section{Introducción}
El transporte público urbano suele enfrentarse a problemas de ineficiencia operativa y a una notable falta de información precisa para los usuarios. En la mayoría de ciudades, los pasajeros desconocen la ubicación exacta de los autobuses en tiempo real, lo que provoca esperas largas e impredecibles, especialmente en horas de alta demanda. Esta incertidumbre no solo afecta la percepción del servicio, sino que también influye en la puntualidad, la comodidad y la planificación diaria de los usuarios, quienes deben ajustarse a horarios aproximados y confiar en estimaciones que rara vez reflejan las condiciones reales del sistema. En los últimos años, diversos estudios han demostrado que los sistemas de monitoreo de flota y seguimiento en tiempo real pueden reducir significativamente los tiempos de espera y mejorar la experiencia del usuario, ya que proporcionan una visión más clara de la operación de los vehículos. Además, investigaciones sobre soluciones de rastreo con datos precisos sugieren que contar con información exacta sobre la llegada de los autobuses aumenta la confianza de los pasajeros y fomenta el uso continuo del transporte público como alternativa sostenible.

Para contextualizar mejor esta problemática, es útil considerar algunas estadísticas globales. Según informes de la Organización Mundial de la Salud (OMS) y la Unión Internacional de Transporte Público (UITP), alrededor del 50\% de la población urbana en países en desarrollo depende del transporte público para sus desplazamientos diarios. Sin embargo, en muchas ciudades latinoamericanas, como Bogotá o Ciudad de México, los usuarios reportan tiempos de espera promedio de 20 a 30 minutos, lo que no solo genera frustración, sino que también contribuye a la congestión vehicular al incentivar el uso de automóviles particulares. Esta situación se agrava en horas pico, donde la falta de información en tiempo real puede llevar a decisiones apresuradas, como caminar largas distancias o esperar en paradas incorrectas. Estudios realizados en Europa y Estados Unidos indican que la implementación de sistemas de rastreo GPS puede reducir estos tiempos de espera en un 30-40\%, además de mejorar la puntualidad general del servicio y reducir las emisiones de carbono al disminuir la circulación innecesaria de vehículos.

La evolución histórica de estas tecnologías también resulta reveladora. Desde los primeros sistemas de radiofrecuencia en los años 80, pasando por el auge de los GPS en los 2000, hasta las soluciones actuales basadas en IoT y big data, el rastreo vehicular ha avanzado considerablemente. Inicialmente, estos sistemas eran costosos y limitados a flotas corporativas, pero el abaratamiento de los dispositivos móviles y la proliferación de APIs abiertas han democratizado el acceso. Hoy en día, plataformas como Google Maps o Moovit ofrecen estimaciones básicas, pero carecen de la integración directa con operadores locales, lo que limita su precisión en contextos urbanos complejos. UrbanTracker se posiciona en esta línea evolutiva, aprovechando tecnologías maduras para ofrecer una solución accesible y adaptable a realidades locales.

Ante este panorama, surge la necesidad de implementar herramientas tecnológicas que permitan modernizar y dinamizar la gestión de los sistemas de transporte urbano. Es en este contexto donde aparece UrbanTracker, una solución integral de geolocalización diseñada para ofrecer información en tiempo real sobre la ubicación de los vehículos y apoyar la administración eficiente de la flota. El objetivo principal de UrbanTracker es permitir que los usuarios consulten fácilmente las rutas disponibles y visualicen, en un mapa interactivo, la posición actualizada de los autobuses que se encuentran en servicio. De esta forma, las personas pueden planear sus trayectos con mayor precisión y adaptarse a las condiciones reales del entorno.

Para lograr este funcionamiento, el sistema integra servicios de mapas, tecnologías GPS y protocolos de comunicación en tiempo real, como WebSockets o MQTT, que permiten la transmisión continua de las coordenadas de cada vehículo. Estos mecanismos garantizan que la información mostrada en la plataforma sea dinámica, confiable y lo suficientemente rápida como para reflejar de manera fiel el desplazamiento de los autobuses. Según la especificación de requisitos, UrbanTracker incluye funcionalidades clave que abarcan tanto a los usuarios como a los operadores: consulta de rutas, visualización de vehículos activos en el mapa, autenticación de conductores para iniciar recorridos, y un módulo administrativo que permite gestionar rutas, conductores y vehículos desde una interfaz centralizada y sencilla de utilizar.

Además de estos beneficios funcionales, UrbanTracker aporta valor en múltiples dimensiones. Para los usuarios finales, significa una reducción tangible en el estrés diario, permitiendo actividades como leer un libro mientras esperan o coordinar citas con mayor exactitud. Para los administradores, implica una herramienta de toma de decisiones basada en datos reales, facilitando la optimización de rutas, la asignación de recursos y la respuesta rápida a incidentes. En un nivel más amplio, contribuye a la sostenibilidad urbana al promover el uso del transporte público y reducir la dependencia de vehículos privados, lo que se alinea con objetivos globales como los de la Agenda 2030 de las Naciones Unidas para el desarrollo sostenible.

El presente documento profundiza en los componentes que conforman la plataforma, el marco teórico que respalda su diseño, la metodología de desarrollo empleada y los resultados obtenidos durante su implementación. Asimismo, se presentan las conclusiones generadas a partir del proceso de construcción de UrbanTracker y su relevancia como herramienta innovadora para mejorar la movilidad en entornos urbanos.
La evolución tecnológica en el rastreo vehicular ha sido impulsada por avances en sensores, conectividad y procesamiento de datos. Desde los primeros sistemas basados en radiofrecuencia en los años 80, que requerían infraestructura dedicada y eran costosos, hemos pasado a soluciones móviles integradas en smartphones. La llegada de GPS preciso y redes celulares ha democratizado el acceso, permitiendo que incluso operadores pequeños implementen sistemas efectivos. Sin embargo, desafíos como la precisión en entornos urbanos densos y la dependencia de baterías móviles persisten, motivando innovaciones continuas en algoritmos de localización y optimización energética.

UrbanTracker se posiciona en esta trayectoria evolutiva, aprovechando tecnologías maduras como MQTT y Spring Boot para ofrecer una solución robusta y escalable. Su diseño modular permite adaptaciones a diferentes contextos, desde flotas escolares hasta servicios de entrega urbana, demostrando versatilidad en aplicaciones reales.
La evolución tecnológica en el rastreo vehicular ha sido impulsada por avances en sensores, conectividad y procesamiento de datos. Desde los primeros sistemas basados en radiofrecuencia en los años 80, que requerían infraestructura dedicada y eran costosos, hemos pasado a soluciones móviles integradas en smartphones. La llegada de GPS preciso y redes celulares ha democratizado el acceso, permitiendo que incluso operadores pequeños implementen sistemas efectivos. Sin embargo, desafíos como la precisión en entornos urbanos densos y la dependencia de baterías móviles persisten, motivando innovaciones continuas en algoritmos de localización y optimización energética.

UrbanTracker se posiciona en esta trayectoria evolutiva, aprovechando tecnologías maduras como MQTT y Spring Boot para ofrecer una solución robusta y escalable. Su diseño modular permite adaptaciones a diferentes contextos, desde flotas escolares hasta servicios de entrega urbana, demostrando versatilidad en aplicaciones reales. La integración de mapas interactivos y notificaciones en tiempo real no solo mejora la experiencia del usuario, sino que también facilita la toma de decisiones operativas, reduciendo tiempos de respuesta y aumentando la eficiencia general del sistema.

Además, la plataforma incorpora medidas de seguridad avanzadas para proteger la privacidad de los datos, cumpliendo con regulaciones internacionales como GDPR y leyes locales de protección de datos. Esto asegura que la información de ubicación se maneje de manera ética y responsable, fomentando la confianza de los usuarios y operadores por igual.
En resumen, UrbanTracker no solo resuelve un problema técnico, sino que contribuye a una transformación más amplia en la movilidad urbana, promoviendo sistemas de transporte más eficientes, sostenibles y centrados en el usuario. Su implementación exitosa valida el potencial de las tecnologías modernas para mejorar la calidad de vida en entornos urbanos complejos.

Además, el sistema se adapta a contextos diversos, desde ciudades grandes con alta densidad de tráfico hasta municipios más pequeños con recursos limitados. La modularidad de UrbanTracker permite personalizaciones según las necesidades locales, como integración con sistemas de pago electrónico o alertas de emergencias. Esta flexibilidad asegura que el proyecto no sea una solución estática, sino un marco evolutivo que crece con las demandas de la sociedad moderna.

Finalmente, UrbanTracker representa un paso adelante en la democratización de la tecnología para el transporte público, haciendo que herramientas avanzadas estén al alcance de operadores de cualquier tamaño. Al combinar accesibilidad con innovación, el sistema sienta las bases para un futuro donde la movilidad urbana sea más inteligente, equitativa y respetuosa con el medio ambiente.