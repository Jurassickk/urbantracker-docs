\section{Introducción}

¿Quién no llegó alguna vez a una parada de bus sin la más mínima idea de si el vehículo ya pasó o viene en camino? Me sucede tan a menudo que dejé de contar. Pero esta no es una cuestión puramente personal: amigos, familiares, colegas y, en general, todos mis conocidos, en una o en otra etapa, se enfrentaron a la misma frustración. Uno de los grandes problemas de nuestro transporte en la ciudad es la falta de información razonable. Los horarios publicados generalmente son aproximados, y nosotros, los usuarios, quedamos varados en un estado de incertidumbre. Un mal cálculo en la hora de llegada puede hacer que seas tarde para una cita, te pierdas una clase, o simplemente añadir estrés innecesario a tu día. Pero esto no es lo peor: esta frustración no es un problema mínimos; incluye tiempo perdido, disminución de la productividad, y una percepción negativa del transporte público. Pero un estudio reciente evidencia que los usuarios que tienen acceso a datos precisos sobre cuánto tiempo le queda antes que llegue el bus se sienten más a gusto y tienen más probabilidades de usarlo nuevamente y con frecuencia. Esto, a su vez, reduce tráfico y emisiones.

¿Cuál es la magnitud de este problema? Para tener una idea general, vale la pena considerar las cifras a nivel mundial. La OMS y la UITP aseguran que, en los países en vías de desarrollo, casi la mitad de la población urbana se desplaza diariamente utilizando transporte público. Sin embargo, en ciudades como Bogotá o Ciudad de México, los usuarios reportaron tiempos de espera promedio entre 20 y 30 minutos. La falta de una estimación clara impulsa a muchos a utilizar el vehículo privado, lo que genera congestión y agrava el caos en las horas pico. En cambio, los estudios en Europa y Estados Unidos descubren que la implementación de sistemas de GPS reduce la espera en un 40%, aumenta la puntualidad y previene los desplazamientos innecesarios. Lo que es aún más indicativo, es la historia de clientes de APCO y su experiencia con estas tecnologías: desde sistemas analógicos de radio caros que solo las grandes empresas podían pagar hasta GPS masivos en los años 2000 y soluciones IoT basadas en big data y dispositivos móviles baratos. Hoy, aunque las plataformas de información de tráfico más potentes, como Google Maps, ofrezcan estimaciones, estas son imprecisas y desactualizadas al no integrarse con las rutas locales. Es aquí donde aparece UrbanTracker: una plataforma que incorpora tecnologías conocidas para ofrecer una solución más precisa, económica y adaptable. En este sentido, se hace indispensable contar con herramientas modernas para llevar la administración del transporte urbano a una nueva etapa. UrbanTracker es la respuesta a esta situación; es una plataforma de geolocalización y seguimiento en tiempo real diseñada para facilitar la administración de flotas empresariales. En otras palabras, se trata de un sistema que permite al cliente promedio verificar rutas y horas de operación, ver dónde se encuentran los buses en tiempo real y planificar geográficamente su ruta de acuerdo con este enfoque.

Para lograrlo, integra mapas digitales, GPS y protocolos ligeros de comunicación como MQTT o WebSockets, los cuales garantizan un flujo continuo y estable de coordenadas. La plataforma incorpora funcionalidades como consulta de rutas, visualización en vivo, autenticación para conductores y un módulo administrativo que centraliza la operación sin complejidad innecesaria. UrbanTracker agrega mucho más que un beneficio directo al usuario. Disminuye la incertidumbre diaria, permite planificar el día y evita esperar horas que parecen eternas. Para operadores y administradores, es un soporte para tomar decisiones basadas en datos reales: optimizar rutas, reasignar recursos y detectar incidentes de manera rápida y eficiente. A nivel ciudad, UrbanTracker promueve la movilidad sostenible, uno de los objetivos de desarrollo sostenible declarados por la ONU en su Agenda 2030. Este documento describe los componentes de UrbanTracker, el marco teórico, la metodología aplicada, los resultados y las conclusiones. Pero antes de profundizar en el lado técnico, considero relevante documentar una historia corta y simple. Hace algunos meses, un amigo llamado Carlos me invitó al almuerzo al centro de Bogotá. Él salió desde Chapinero en TransMilenio, y yo fui en carro, ya que no quería generar la incertidumbre de siempre de los buses. Cuando llegué, él ya estaba sentado, pero bastante molesto. Carlos compartió que esperó cerca de 25 minutos en la parada, ya que confundió un bus con otro: el que ingresó creyendo que era su ruta terminó siendo el 123 en lugar del 456, lo que, con un simple error, le generó casi media hora de retraso en el plan y un buen nivel de estrés. Es así como inició el desarrollo de UrbanTracker.

Lo mismo le ha pasado a mi hermana, estudiante universitaria. A pesar de salir con tiempo, muchas veces llega tarde porque en la parada no saben si el bus está cerca o si ya pasó. No es una simple molestia: afecta asistencia a clases, rendimiento, estado de ánimo y hasta la percepción general del transporte público. Miremos si extrapolamos un poco, el banco Mundial estima la espera en ciudades como Bogotá entre 15 y 25 minutos al día. Una persona que use transporte público en un día normal hasta dos horas esperando. Multiplíquelo por los millones de usuarios diarios y el impacto económico está claro. La Universidad de Los Andes estima que la incertidumbre en movilidad en Colombia cuesta más de 1001 USD. Hay un impacto ambiental significativo. La gente no usa transporte público porque sabe que no llega a tiempo, entonces usa su carro. En Bogotá, entre el 70% de los viajes en hora pico son en carro, lo que significa una fuente significante de contaminación atmosférica de la ciudad. UrbanTracker es la respuesta directa a esto. UrbanTracker no es solo una aplicación más: es la solución al origen del problema, con información precisa en estos momentos. Puede ver dónde está su bus y cuánto falta para que llegue, los conductores pueden mejorar su operación y los administradores obtienen datos para saber cómo gestionar las rutas en tiempo real.

Un futuro en el que no haya necesidad de adivinar o confiar demasiado. Donde abre su teléfono, ve el movimiento del autobús en vivo y toma decisiones informadas sobre su viaje. Ese futuro no es hipotético; está a su alcance con tecnología existente y asequible. El potencial transformador de UrbanTracker no se limita a lo técnico. También es social, económico y ambiental. Fomenta el uso del transporte público, disminuye el tráfico y la huella de carbono y fortalece la vida urbana. En un mundo donde la movilidad es un desafío en crecimiento, la propuesta como UrbanTracker es un paso real en el camino hacia ciudades más inteligentes, eficaces y humanas.