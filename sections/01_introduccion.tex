\section{Introducción}
El transporte público urbano a menudo presenta una problemática por una mala experiencia del usuario principalmente marcada por la falta de información sobre las rutas y ubicación de los vehículos de transporte, lo que resulta en falta de interés y de información precisa.

UrbanTracker, un Sistema de Geolocalización de Rutas de Transporte, aborda esta problemática ofreciendo una solución accesible que utiliza los dispositivos móviles de los conductores como GPS para enviar datos de geolocalización para mostrar una visualización en tiempo real para los usuarios y la actualización de sus posiciones. Se requiere la integración de algunas tecnologías IoT (Internet de las Cosas) para la comunicación eficiente e ágil para una baja latencia, utilizando el protocolo MQTT para la comunicación.