\section{Introducción}

El transporte público urbano enfrenta desafíos significativos en la experiencia del usuario, caracterizados por la ausencia de información actualizada sobre rutas y ubicaciones de vehículos. Esta falta de transparencia genera incertidumbre, aumenta los tiempos de espera y reduce la confianza en el sistema, lo que a su vez disminuye el uso del transporte público en favor de alternativas privadas. En ciudades densamente pobladas, donde el tráfico y la congestión son comunes, los usuarios dependen de información precisa para planificar sus desplazamientos, pero las aplicaciones tradicionales a menudo fallan en proporcionar datos en tiempo real debido a limitaciones en la transmisión de datos.

Por ejemplo, en ciudades como Bogotá, Colombia, donde el sistema TransMilenio transporta millones de pasajeros diarios, los usuarios reportan tiempos de espera promedio de 15-20 minutos en horas pico, con incertidumbre sobre la ubicación exacta de los buses. En Medellín, el Metroplús enfrenta similar problemática, donde la falta de información en tiempo real resulta en aglomeraciones en estaciones y reducción del 25% en la eficiencia del servicio. Estudios locales indican que la desinformación contribuye a una reducción del 20-30% en la utilización del transporte público, exacerbando problemas ambientales como el aumento de emisiones de CO2 por el uso de vehículos particulares, y de movilidad urbana con congestión vial crónica.

Además, los administradores de flotas carecen de herramientas para monitorear eficientemente las rutas, lo que impide optimizaciones en la asignación de vehículos y la gestión de emergencias. En contextos urbanos latinoamericanos, donde los presupuestos municipales son limitados, la inversión en sistemas de geolocalización tradicionales resulta prohibitiva, dejando a las entidades de transporte sin capacidad para responder a demandas dinámicas como desvíos por obras viales o eventos masivos.

UrbanTracker surge como una solución integral para abordar esta desinformación, proporcionando acceso en tiempo real a la ubicación y rutas de los vehículos tanto para administradores como para el público general. Como Sistema de Geolocalización de Rutas de Transporte, enfrenta desafíos técnicos clave, como la selección de un protocolo de comunicación capaz de gestionar un alto volumen de mensajes GPS de manera continua, eficiente y con bajo costo operativo. La solución propuesta utiliza dispositivos móviles de los conductores como sensores GPS, enviando datos de geolocalización para una visualización en tiempo real en plataformas web y móviles.

Para lograr una comunicación eficiente con baja latencia, se integra tecnologías IoT, específicamente el protocolo MQTT, que permite una transmisión ligera y escalable de datos. Este enfoque no solo resuelve la problemática de la desinformación, sino que también promueve un transporte público más confiable y accesible, mejorando la calidad de vida en entornos urbanos.

El principal propósito del presente documento es validar e informar por qué se ha hecho la selección del protocolo MQTT (protocolo de envío ligero y eficiente) para la eficiente comunicación en tiempo real en la arquitectura de UrbanTracker, basándose en el exhaustivo análisis de investigación de trabajos previos en el campo de las tecnologías IoT (Internet de las Cosas) y la geolocalización.

\begin{figure}[h]
\centering
\includegraphics[width=0.4\textwidth]{./graphics/Imagen-11.jpg}
\caption{Arquitectura general del sistema UrbanTracker, mostrando el flujo de datos desde el dispositivo móvil del conductor hasta los usuarios finales mediante MQTT.}
\end{figure}

Esta arquitectura, como se ilustra en la Figura 1, permite una comunicación bidireccional eficiente, donde los datos GPS se transmiten de manera continua sin sobrecargar los recursos del dispositivo móvil. En un contexto urbano moderno, donde la movilidad sostenible es crucial, sistemas como UrbanTracker no solo mejoran la accesibilidad al transporte público, sino que también facilitan la planificación urbana inteligente, reduciendo emisiones y optimizando el flujo vehicular mediante la reducción de tiempos de espera y la minimización de rutas ineficientes.