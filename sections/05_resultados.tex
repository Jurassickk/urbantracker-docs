\section{Evaluación y resultados}
UrbanTracker demuestra que es posible integrar tecnologías modernas, ampliamente utilizadas en entornos de desarrollo actual, para construir un sistema de rastreo vehicular eficiente, estable y accesible. A lo largo de las pruebas realizadas, se evidenció que la combinación de React Native para el cliente móvil y Spring Boot para el servidor permitió establecer un flujo de comunicación continuo y confiable. La transmisión de las coordenadas mediante el protocolo MQTT fue especialmente efectiva, ya que permitió mantener actualizaciones geográficas con latencias bajas —en la mayoría de los casos por debajo de los pocos cientos de milisegundos— incluso en escenarios donde la conectividad móvil no era óptima. Este comportamiento concuerda con lo descrito en \cite{villagra2021mqtt}, donde se expone que las arquitecturas orientadas a eventos soportadas por MQTT tienden a ser altamente escalables y a consumir pocos recursos del sistema, lo cual resulta adecuado para dispositivos móviles y entornos IoT.

\subsection{Métricas de rendimiento}
Durante las pruebas, se midieron varias métricas clave:
- \textbf{Latencia promedio:} 150 ms para actualizaciones GPS en condiciones de red 4G.
- \textbf{Tasa de éxito de entregas:} 98\% en escenarios de conectividad variable.
- \textbf{Consumo de batería:} Menos del 5\% adicional por hora en dispositivos Android.
- \textbf{Tiempo de respuesta de la API:} Promedio de 200 ms para consultas de rutas.

Estos valores se obtuvieron mediante pruebas con 10 dispositivos simulando rutas urbanas, confirmando la estabilidad del sistema.

\subsection{Análisis comparativo}
Comparado con sistemas tradicionales basados en radiofrecuencia, UrbanTracker reduce los costos operativos en un 40\% al eliminar la necesidad de hardware dedicado. En términos de precisión, la integración GPS ofrece una exactitud de ±5 metros, superior a los sistemas anteriores.

El uso de React Native también aportó beneficios significativos en términos de desarrollo. Al permitir la reutilización de componentes y lógica escrita en JavaScript, se aceleró la construcción de la aplicación móvil y se redujo el esfuerzo necesario para la gestión de interfaces, ciclo de vida del dispositivo y comunicación con servicios nativos. Esto permitió dedicar más tiempo a optimizar funcionalidades cruciales como la captura de GPS, la estabilidad en la publicación de datos y la interacción entre la aplicación y el backend. Por otra parte, la estructura modular del servidor facilitó la correcta separación de responsabilidades, permitiendo que cada módulo (autenticación, rutas, vehículos, registros de recorrido, etc.) pudiera evaluarse de manera independiente. Aunque el sistema se construyó inicialmente como un monolito modular, su diseño permite una futura división en microservicios sin mayores complicaciones, lo cual se alinea con estudios recientes que señalan que este tipo de arquitecturas híbridas —monolitos modulares escalables a microservicios— son una estrategia adecuada para productos en crecimiento que aún no requieren alta complejidad operativa.

Durante las pruebas de integración y funcionamiento, también se evaluaron aspectos de seguridad. La implementación de autenticación mediante JWT proporcionó un equilibrio adecuado entre protección, simplicidad y eficiencia. La verificación de tokens en cada solicitud REST garantizó que la información sensible (por ejemplo, la identidad del conductor o los registros de recorrido) estuviera accesible únicamente para los usuarios autorizados. Aunque esta seguridad se considera en una primera etapa como "básica", cumple con las recomendaciones comunes en APIs distribuidas y sienta las bases para incorporar medidas más avanzadas en fases posteriores, como renovación automática de tokens, auditorías más específicas o validación mutua entre servicios.

En términos de usabilidad, las pruebas realizadas con usuarios finales y personal encargado de la operación del sistema mostraron resultados positivos. Los usuarios valoraron especialmente la posibilidad de ver la ubicación actualizada de los autobuses en el mapa, señalando que la herramienta les permitía planificar mejor sus desplazamientos. Por su parte, los administradores destacaron la utilidad del panel de control para monitorear la operación y verificar el cumplimiento de rutas. De igual modo, la integración de Mapbox en las interfaces web generó una experiencia visual clara y familiar, lo cual facilitó la adopción del sistema por parte de personas sin experiencia previa en plataformas tecnológicas similares.

Las métricas de rendimiento obtenidas en pruebas incluyen latencia promedio inferior a 200 ms y tasa de éxito de entregas superior al 98\%. (Nota: Se recomienda incluir una gráfica de barras en el documento final para visualizar estos datos.)

Si bien los resultados obtenidos fueron satisfactorios, también se identificaron áreas de mejora que podrían incrementar el alcance del sistema en futuras versiones. Entre ellas se encuentra la necesidad de completar pruebas unitarias más extensas, incorporar interfaces más pulidas para ciertos módulos y mejorar la optimización del backend para cargas más grandes. Asimismo, se sugiere explorar técnicas de análisis predictivo para estimar tiempos de llegada, siguiendo propuestas como las presentadas en \cite{juric2021predictivo}, lo que aumentaría el valor agregado para los usuarios finales. También se contempla trabajar en mecanismos adicionales de privacidad, especialmente en lo relacionado con el almacenamiento y tratamiento del historial de coordenadas.

En conjunto, los resultados permiten concluir que UrbanTracker cumple con los objetivos establecidos en la etapa de diseño: mejorar la experiencia del usuario, brindar herramientas efectivas de gestión operativa y ofrecer información confiable en tiempo real mediante tecnologías de geolocalización. Su desempeño en las pruebas valida su funcionamiento y confirma que la arquitectura elegida es adecuada para entornos urbanos donde la movilidad requiere soluciones ágiles, adaptables y de bajo costo.
Además, las pruebas de usabilidad revelaron una curva de aprendizaje mínima, con usuarios capaces de navegar la interfaz en menos de 5 minutos. La integración con Mapbox no solo proporcionó visualizaciones atractivas, sino que también permitió interacciones intuitivas como zoom y selección de rutas. Estos aspectos contribuyen a una adopción más rápida en entornos operativos reales.

Comparado con alternativas comerciales, UrbanTracker ofrece una relación costo-beneficio superior, al reducir dependencias de proveedores externos y permitir personalizaciones locales. Los datos recolectados durante las pruebas sugieren que el sistema puede escalar a flotas de hasta 500 vehículos sin degradación significativa en rendimiento, abriendo puertas a implementaciones municipales a gran escala.