\section{Marco teórico y trabajos relacionados}
Aportes clave seleccionados del thesaurus y su relación con UrbanTracker:

• DESARROLLO DE UN SISTEMA DE LOCALIZACIÓN BASADO EN GPS E IOT: UN ESTUDIO DE CASO EN QUITO (2025). Fuente: Google Sschoolar. https://www.investigacionistct.ec/ojs/index.php/investigacion_tecnologica/article/view/165

• SISTEMA DE GEOLOCALIZACION DE VEHICULOS RECOLECTORES DE BASURA APLICANDO INTERNET DE LAS COSAS (2025). Fuente: Google Sschoolar. https://repositorio.upea.bo/jspui/handle/123456789/89

• Aplicación web para el control de desviaciones de rutas en el transporte público mediante IOT (2025). Fuente: Google Sschoolar. https://www.dspace.espol.edu.ec/handle/123456789/65811

• Aplicación del protocolo MQTT y recolección de datos para aplicaciones IoT (2025). Fuente: Google Sschoolar. https://repositorio.unitec.edu/items/2b823853-4bb8-4fd7-887a-d93f122933c5

• Arquitectura orientada a eventos sobre protocolo MQTT (2025). Fuente: Google Sschoolar. https://sedici.unlp.edu.ar/handle/10915/130301

En conjunto, estos trabajos recomiendan una arquitectura de tres capas con GPS en el vehículo, un pub/sub —preferentemente MQTT— y cliente web para visualización. UrbanTracker adopta este enfoque y lo aplica a un contexto local.

MARCO DE REFERENCIA — MQTT: Según «Arquitectura orientada a eventos sobre protocolo MQTT» (SEDICI–UNLP, 2025), el paradigma publicación/suscripción de MQTT reduce latencia y uso de ancho de banda en telemetría vehicular.