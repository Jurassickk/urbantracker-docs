\section{Marco teórico y trabajos relacionados}

Al iniciar la revisión de soluciones para rastreo en tiempo real, se observa que estos sistemas combinan dispositivos capaces de captar coordenadas GPS con mecanismos de transmisión continua de datos. Esta integración posibilita visualizar posiciones en mapas, emitir alertas y alimentar modelos de análisis instantáneo. En el contexto urbano, dichas herramientas se han convertido en componentes esenciales para optimizar la movilidad, ya que ofrecen una visión clara del comportamiento de una flota y permiten anticipar retrasos.

En el ámbito del transporte público, el crecimiento del ecosistema IoT ha impulsado la incorporación de sensores, módulos GPS y protocolos ligeros de mensajería que entregan información constante sin necesidad de infraestructura costosa. Por ejemplo, en \cite{karne2022iot} se presenta una propuesta basada en GPS y MQTT para transmitir posiciones de buses en tiempo real, demostrando que es posible obtener un seguimiento estable y económico incluso en redes con conectividad variable. El estudio muestra que el uso de sensores accesibles puede reducir gastos operativos entre 20 y 30 %, aunque requiere ajustes de calibración para evitar imprecisiones en zonas densamente urbanizadas.

La arquitectura del sistema resulta un elemento crítico cuando el número de vehículos crece o la frecuencia de actualización se incrementa. A medida que el flujo de datos aumenta, es necesario adoptar estrategias capaces de procesar grandes volúmenes de trayectorias sin degradar el rendimiento. Según \cite{cervantes2022arquitectura}, una arquitectura distribuida puede manejar miles de actualizaciones simultáneas con alta tolerancia a fallos, lo que la convierte en una opción adecuada para soluciones escalables. No obstante, incorpora desafíos como la complejidad de sincronización y la posible variabilidad en la latencia. Por ello, elegir entre un backend monolítico modular o una arquitectura basada en microservicios depende del contexto. En \cite{nugraha2023api} se evidencia que APIs construidas con microservicios y WebSocket pueden mejorar la estimación de tiempos de llegada al integrar datos externos como tráfico; sin embargo, este enfoque incrementa la complejidad operativa y los costos de mantenimiento entre 15 y 25 %. En este sentido, UrbanTracker optó por iniciar con un monolito modular, permitiendo una transición progresiva hacia microservicios sin reescrituras completas y manteniendo un balance entre simplicidad y flexibilidad.

En cuanto a las aplicaciones móviles, el auge de los frameworks multiplataforma ha favorecido la reducción de tiempos y costos de desarrollo. Soluciones como React Native y Flutter permiten obtener apps nativas desde un único código. Diversos análisis señalan que la elección depende de la experiencia del equipo y de las necesidades de integración con servicios externos. React Native destaca por su madurez, el extenso ecosistema JavaScript y la facilidad de incorporar librerías cartográficas, aunque ofrece menos control sobre APIs nativas avanzadas en comparación con Flutter. Por estas razones, UrbanTracker adoptó React Native para la app del conductor, priorizando compatibilidad y buen rendimiento en dispositivos Android, predominantes en el transporte urbano. Esta decisión coincide con tendencias globales donde React Native se impone en aplicaciones productivas, mientras Flutter destaca más en interfaces altamente interactivas.

Respecto a la comunicación en tiempo real, UrbanTracker emplea MQTT como protocolo pub/sub ligero. Como describe \cite{villagra2021mqtt}, las arquitecturas orientadas a eventos respaldadas por MQTT suelen ser altamente escalables y consumir pocos recursos. Aunque en ciertas comparativas WebSocket presenta mejor rendimiento en CPU o memoria, MQTT tiende a ofrecer mayor estabilidad en escenarios donde la conexión móvil fluctúa. En situaciones con cobertura irregular, MQTT mantiene enlaces persistentes con un overhead reducido, un aspecto clave para vehículos que atraviesan zonas sin señal. De todos modos, requiere configurar adecuadamente el broker para evitar saturaciones cuando existe alta concurrencia. Alternativas como AMQP brindan mayor robustez transaccional, aunque a costa de una mayor complejidad.

La seguridad también es un componente central. Según \cite{shukla2025oauth}, mecanismos como OAuth2 y JWT proporcionan autenticación sólida en sistemas distribuidos, garantizando que solo usuarios autorizados accedan a las APIs. En el contexto MQTT, estudios como \cite{aguirre2020mqtt} recomiendan emplear capas de cifrado que aseguren los mensajes frente a posibles interceptaciones. UrbanTracker implementa autenticación JWT en su backend siguiendo estas buenas prácticas. Aun así, JWT presenta limitaciones en procesos de revocación inmediata, lo que se considera un riesgo menor para la escala inicial del proyecto.

Además del plano técnico, resulta relevante analizar el impacto social y urbano. En varias ciudades, estos sistemas han contribuido a disminuir la congestión al optimizar rutas en tiempo real, como demuestran casos en Londres y Singapur. Sin embargo, también plantean retos éticos relacionados con la privacidad de datos, que pueden mitigarse mediante políticas claras de retención y anonimización. UrbanTracker incorpora estos principios al diseñar interfaces que minimizan la exposición de información sensible.

En conjunto, la revisión teórica respalda las decisiones arquitectónicas del proyecto: uso de servicios cartográficos y geolocalización, desarrollo móvil con React Native, backend en Spring Boot con APIs REST y mensajería MQTT para tiempo real, además de mecanismos de seguridad basados en JWT/OAuth2. La literatura confirma que esta combinación es adecuada para sistemas de seguimiento vehicular en entornos urbanos con niveles variables de conectividad y necesidades de escalabilidad. También revela áreas de mejora futura, como el uso de IA para predicciones más precisas.

Un aspecto complementario es el manejo de datos en streaming. Investigaciones como \cite{fernandez2023bigdata} resaltan la importancia de procesar flujos en tiempo real para mejorar la estimación de tiempos de llegada. UrbanTracker se beneficia de arquitecturas compatibles con bases NoSQL como MongoDB para manejar volúmenes fluctuantes de ubicaciones, aunque se optó por PostgreSQL debido a su solidez en consultas espaciales y consistencia transaccional.

Experiencias europeas, como la de Londres, indican que la incorporación de datos meteorológicos y de tráfico puede reducir errores de predicción entre 20 y 30 %. UrbanTracker está preparado para integrar estos módulos en fases posteriores gracias a su arquitectura modular.

En cuanto a usabilidad, estudios como \cite{gomez2024ux} subrayan la importancia de interfaces adaptativas para usuarios con niveles diversos de alfabetización digital. Las pruebas iniciales del sistema muestran que su diseño intuitivo favorece la adopción incluso entre adultos mayores, siguiendo tendencias internacionales de inclusión en servicios públicos digitales.

La seguridad en comunicaciones IoT continúa evolucionando. Aunque CoAP presenta alternativas de ligereza, MQTT sigue sobresaliendo en estabilidad para dispositivos móviles. La revisión de \cite{lopez2022security} respalda el uso de JWT en sistemas distribuidos y sugiere incorporar cifrado end-to-end en etapas futuras.

Finalmente, el impacto social de estas tecnologías es considerable. Iniciativas en ciudades emergentes, como Curitiba en Brasil, evidencian que la disponibilidad de información en tiempo real puede incrementar el uso del transporte público hasta en un 25 %, reduciendo emisiones y descongestión vial. UrbanTracker se integra en este panorama, aportando a los debates sobre sostenibilidad y equidad en la movilidad urbana.