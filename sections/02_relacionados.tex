\section{Marco teórico y trabajos relacionados}
Los sistemas de rastreo en tiempo real combinan dispositivos capaces de obtener posiciones GPS con canales de comunicación diseñados para transmitir datos de manera casi instantánea, lo que permite mostrar la ubicación de vehículos en mapas, generar alertas o alimentar algoritmos de análisis. Este tipo de soluciones se ha convertido en una herramienta fundamental para mejorar la movilidad urbana, ya que permiten conocer con precisión el estado operativo de una flota y anticipar posibles retrasos. En el ámbito del transporte público, el avance de tecnologías asociadas al Internet de las Cosas (IoT) ha impulsado la implementación de sensores, módulos GPS y plataformas ligeras de mensajería que permiten ofrecer información continua sin necesidad de infraestructuras complejas. Por ejemplo, en \cite{karne2022iot} se presenta un sistema basado en IoT que emplea dispositivos GPS instalados en autobuses y un esquema de comunicación mediante MQTT para transmitir coordenadas en tiempo real, demostrando que es posible lograr un seguimiento estable y económico incluso en redes de conectividad limitada. Este estudio destaca cómo la integración de sensores de bajo costo puede reducir los costos operativos en un 20-30\% comparado con sistemas tradicionales, aunque requiere una calibración inicial para evitar errores de precisión en áreas urbanas densas.

La arquitectura del sistema juega un papel crucial cuando aumenta el número de vehículos o cuando se requiere una frecuencia de actualización elevada. A medida que el volumen de datos crece, se vuelve imprescindible adoptar enfoques que permitan procesar grandes cantidades de trayectorias sin degradar el rendimiento. Según lo planteado en \cite{cervantes2022arquitectura}, una arquitectura distribuida es capaz de manejar miles de actualizaciones simultáneas con altos niveles de tolerancia a fallos, lo que la convierte en una opción atractiva para sistemas de transporte con escalabilidad progresiva. Sin embargo, esta ventaja viene con desafíos, como la complejidad en la sincronización de datos y el riesgo de latencias variables en redes distribuidas. En este sentido, la elección entre un backend monolítico modular y una arquitectura de microservicios debe evaluarse según el contexto. Estudios como \cite{nugraha2023api} muestran que las APIs basadas en microservicios, en conjunto con WebSocket, pueden mejorar la calidad de las predicciones de llegada al integrar datos provenientes de diversas fuentes externas, como el tráfico. No obstante, también se reconoce que migrar hacia microservicios introduce complejidades adicionales en el despliegue, el monitoreo y la gestión operativa, lo que puede aumentar los costos de mantenimiento en un 15-25\% según experiencias reportadas. Por esta razón, UrbanTracker adopta inicialmente un monolito modular, que conserva una separación lógica clara y permite evolucionar hacia microservicios sin incurrir en reescrituras extensas, ofreciendo un equilibrio entre simplicidad y flexibilidad.

En lo relacionado con el cliente móvil, los frameworks multiplataforma han ganado popularidad debido a la rapidez que ofrecen en los ciclos de desarrollo. Herramientas como React Native y Flutter permiten construir aplicaciones nativas utilizando una sola base de código, reduciendo costos y tiempos de implementación. Distintos análisis comparativos señalan que la elección suele depender de la experiencia del equipo y de las capacidades de integración requeridas. React Native, basado en JavaScript, destaca por su madurez, por la amplitud de su ecosistema y por la facilidad con la que se puede integrar con librerías de mapas, sensores y servicios de localización. Por ejemplo, su compatibilidad con librerías como react-native-maps facilita la implementación de interfaces intuitivas, aunque puede presentar limitaciones en el acceso a APIs nativas avanzadas comparado con Flutter. En contraste, Flutter ofrece un mejor rendimiento en animaciones complejas, pero requiere un aprendizaje adicional para equipos familiarizados con JavaScript. Por estas razones, y con el fin de agilizar la construcción de la app del conductor, UrbanTracker opta por React Native, garantizando compatibilidad y buen rendimiento en dispositivos Android, que son los más comunes en entornos de transporte público. Esta decisión se alinea con tendencias del mercado, donde React Native domina en aplicaciones de productividad, mientras que Flutter gana terreno en interfaces más interactivas.

En lo correspondiente a la comunicación en tiempo real, UrbanTracker emplea MQTT, un protocolo de mensajería pub/sub que se caracteriza por su ligereza y bajo consumo de recursos. Como demuestran los autores en \cite{villagra2021mqtt}, las arquitecturas guiadas por eventos basadas en MQTT resultan altamente escalables y presentan un consumo reducido tanto en el dispositivo emisor como en el servidor receptor. Comparaciones entre diferentes protocolos evidencian que, aunque WebSocket puede ofrecer una mayor eficiencia en CPU y memoria en ciertos escenarios, MQTT generalmente supera a otros protocolos en estabilidad y uso de datos cuando se trabaja con dispositivos móviles sujetos a variaciones de conectividad. Por instancia, en entornos con cobertura intermitente, MQTT puede mantener conexiones persistentes con menor overhead, lo que resulta crucial para aplicaciones de transporte donde los vehículos pueden pasar por zonas de sombra. Sin embargo, MQTT requiere una configuración cuidadosa del broker para evitar cuellos de botella en escenarios de alta concurrencia, y alternativas como AMQP ofrecen mayor robustez en transacciones críticas, aunque con un costo en complejidad.

Otro aspecto fundamental es la seguridad en la transmisión de la información. Como se expone en \cite{shukla2025oauth}, los protocolos OAuth2 y JWT permiten establecer mecanismos de autenticación sólidos en APIs distribuidas, garantizando que solo usuarios autorizados puedan acceder a los datos. A su vez, en el ámbito específico de MQTT, investigaciones como \cite{aguirre2020mqtt} proponen añadir capas de cifrado sobre el canal de comunicación para mitigar riesgos de interceptación o manipulación de mensajes. UrbanTracker incorpora estas recomendaciones adoptando autenticación mediante JWT en el backend, lo que asegura que los datos de ubicación y las operaciones críticas se mantengan protegidas sin afectar el rendimiento general del sistema. No obstante, es importante considerar que JWT tiene limitaciones en revocación inmediata de tokens, lo que podría requerir implementaciones adicionales como listas negras para escenarios de alta seguridad. En comparación, OAuth2 ofrece mayor flexibilidad en flujos de autorización, pero introduce complejidad en configuraciones simples.

Además de estos componentes técnicos, es relevante analizar el impacto de estas tecnologías en el contexto urbano. Sistemas similares han demostrado reducir la congestión vehicular al optimizar rutas en tiempo real, como se observa en implementaciones en ciudades como Londres o Singapur. Sin embargo, también plantean desafíos éticos, como la privacidad de los datos de ubicación, que deben abordarse mediante políticas claras de retención y anonimización. UrbanTracker considera estos aspectos al diseñar interfaces que minimicen la exposición de datos sensibles y permitan a los usuarios controlar su información.

En síntesis, el marco teórico consultado respalda las decisiones técnicas tomadas para el desarrollo de UrbanTracker: el uso de servicios de mapas y tecnologías de geolocalización, la selección de React Native para el cliente móvil, la implementación de un backend en Spring Boot con APIs REST complementadas por un canal MQTT para transmisión en tiempo real y la adopción de mecanismos de seguridad basados en OAuth2/JWT. Las investigaciones revisadas confirman que esta combinación tecnológica no solo es viable, sino también eficaz para construir sistemas de seguimiento vehicular capaces de operar en entornos urbanos con distintos niveles de conectividad y escalabilidad. Esta revisión exhaustiva no solo valida el enfoque, sino que también identifica oportunidades para futuras mejoras, como la integración de inteligencia artificial para predicciones más precisas.