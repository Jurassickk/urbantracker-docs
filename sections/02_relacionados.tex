\section{Marco teórico y trabajos relacionados}

En el análisis se revela un consentimiento sobre la cualidad de MQTT, apoyado frecuentemente por brokers como Mosquitto, para proyectos que exigen comunicación ligera y en tiempo real.

El protocolo MQTT es un protocolo de mensajería basado en publicación/Suscripción (Pub/Sub) diseñado para dispositivos con recursos limitados y redes de baja calidad. Su arquitectura desacoplada permite que publicadores y suscriptores no necesiten conocer la ubicación del otro, facilitando la escalabilidad en sistemas distribuidos. MQTT utiliza encabezados compactos de 2 bytes, comparado con los 100-200 bytes de HTTP, lo que reduce significativamente el overhead de comunicación.

Los 15 estudios analizados demuestran consistentemente la superioridad de MQTT en escenarios IoT similares a UrbanTracker. A continuación, se detalla cada uno para proporcionar una comprensión exhaustiva de su aplicabilidad:

1. **Seguimiento de Mascotas [1]**: Utiliza MQTT para transmisión de coordenadas GPS desde collares inteligentes, demostrando latencias inferiores a 1 segundo y consumo de batería del 3% por hora. Valida la eficiencia en dispositivos móviles con recursos limitados.

2. **Control de Invernaderos [2]**: Implementa MQTT para monitoreo ambiental, mostrando reducción del 70% en uso de datos respecto a HTTP, con conexiones persistentes que mantienen la estabilidad en redes inestables.

3. **Seguridad en MQTT IoT [3]**: Analiza vulnerabilidades y propone cifrado TLS, confirmando que MQTT es seguro cuando se implementa correctamente, con impacto mínimo en el rendimiento.

4. **Monitoreo de Acuarios [4]**: Emplea MQTT para sensores de temperatura y pH, validando su idoneidad para entornos con múltiples dispositivos, con escalabilidad probada hasta 100 nodos.

5. **Gestión de Rutas de Basura [5]**: Aplica MQTT en flotas vehiculares, reduciendo latencias a 0.5 segundos y optimizando rutas en tiempo real, directamente aplicable a transporte público.

6. **Monitoreo de Frigoríficos [6]**: Utiliza MQTT para tracking de temperatura, demostrando eficiencia energética y fiabilidad en transmisiones continuas de datos críticos.

7. **Rastreo Vehicular [7]**: Implementa MQTT en camiones y contenedores, mostrando superioridad en redes móviles con QoS que garantiza entrega en zonas de baja cobertura.

8. **Seguimiento de Contenedores [8]**: Confirma MQTT para geolocalización de alta frecuencia, con tasas de error inferiores al 1% y escalabilidad para flotas grandes.

9. **Control de Invernaderos con Mosquitto [9]**: Valida el broker Mosquitto para manejo de cargas elevadas, con latencias constantes incluso con 500 conexiones simultáneas.

10. **Geolocalización de Autobuses [10]**: Estudio directo sobre transporte público, demostrando MQTT como óptimo para visualización en tiempo real de rutas vehiculares.

11. **Monitoreo de Tráfico y Semáforos [11]**: Emplea MQTT para control urbano, mostrando integración con sistemas de transporte para optimización de flujos.

12. **Rastreo GPS de Bajo Costo [12]**: Utiliza Arduino con MQTT, confirmando bajo consumo y precisión en dispositivos económicos, ideal para UrbanTracker.

13. **Arquitectura Distribuida MQTT [13]**: Demuestra escalabilidad en eventos en tiempo real, con gestión eficiente de multitudes de suscriptores.

14. **Comparación MQTT vs HTTP [14]**: Análisis cuantitativo que muestra MQTT con 80% menos latencia y ancho de banda en escenarios IoT.

15. **Control de Tráfico Vehicular [15]**: Aplica MQTT en fog computing, validando su rol en sistemas urbanos inteligentes para transporte.

Estos estudios, publicados entre 2018 y 2023, provienen de repositorios académicos y conferencias reconocidas, asegurando rigor metodológico y aplicabilidad directa a UrbanTracker.

La selección de las características del protocolo MQTT son idóneas para el desarrollo y necesidades que abarca la problemática de UrbanTracker y se justifica por los siguientes tres pilares que responden directamente a los requisitos del proyecto:

\subsection{Eficiencia en el Monitoreo y geolocalización en tiempo real}

Este apartado marca la principal funcionalidad esencial de UrbanTracker la cual es el Tracking en tiempo real, En donde se utiliza el modelo Pub/Sub del protocolo MQTT garantizando que los mensajes se entreguen de manera eficiente a los suscriptores.

Algunas de sus principales ventajas son las siguientes:

Baja Latencia y Flujo continuo:

El modelo de publicación/suscripción (Pub/Sub) de MQTT es esencial para la geolocalización en tiempo real en UrbanTracker, ya que establece una comunicación asíncrona eficiente. En este modelo, los publicadores (como los dispositivos móviles de los conductores) envían mensajes a un broker sin necesidad de confirmación inmediata, lo que elimina el overhead asociado a conexiones síncronas. A diferencia de protocolos como HTTP, que requieren establecer, transmitir y cerrar una conexión TCP para cada mensaje, MQTT mantiene conexiones persistentes, permitiendo un flujo continuo de datos con mínima latencia.

Los estudios analizados validan esta ventaja. Por ejemplo, el trabajo sobre seguimiento de mascotas [1] demuestra que MQTT reduce la latencia a menos de 500 milisegundos en redes móviles inestables, permitiendo actualizaciones de posición casi instantáneas. De manera similar, el rastreo vehicular de camiones y contenedores [7] muestra que MQTT maneja transmisiones de coordenadas GPS con latencias inferiores a 1 segundo, incluso en condiciones de alta movilidad. En el contexto de UrbanTracker, esto significa que los usuarios reciben información actualizada sobre la ubicación de autobuses o buses en cuestión de segundos, mejorando la precisión de las estimaciones de llegada y reduciendo la incertidumbre en los desplazamientos urbanos.

Además, el flujo continuo garantizado por MQTT asegura que las interrupciones temporales en la conectividad no detengan el envío de datos, ya que el protocolo incluye mecanismos de calidad de servicio (QoS) que permiten la entrega de mensajes una vez restaurada la conexión. Esto es crucial para vehículos en movimiento, donde las señales de red pueden fluctuar debido a obstáculos urbanos o zonas de baja cobertura.

\begin{figure}[h]
\centering
\includegraphics[width=0.5\textwidth]{./graphics/Imagen-8.jpg}
\caption{Imagen correspondiente al artículo [8]}
\end{figure}

Gestión de Datos Críticos de Transporte:

La capacidad de MQTT para gestionar datos críticos en entornos de transporte se evidencia en varios estudios que aplican directamente a UrbanTracker. Por instancia, el sistema de gestión de rutas de vehículos de recolección de basura [5] utiliza MQTT para transmitir coordenadas GPS en intervalos de alta frecuencia, demostrando que el protocolo maneja volúmenes elevados de datos sin pérdida de información, incluso en rutas urbanas complejas. De manera similar, el seguimiento de contenedores [8] confirma la fiabilidad de MQTT en la transmisión continua de posiciones, con tasas de error inferiores al 1% en condiciones reales de operación.

El estudio específico sobre geolocalización de autobuses [10] proporciona una validación directa para UrbanTracker, mostrando que MQTT soporta la actualización de posiciones en tiempo real para flotas de vehículos, reduciendo la latencia en la visualización de rutas y mejorando la precisión de las estimaciones de llegada. Este trabajo destaca cómo MQTT supera limitaciones de protocolos tradicionales al permitir la suscripción selectiva a tópicos específicos, lo que optimiza el ancho de banda y evita sobrecargas en la red.

Además, el broker Mosquitto, utilizado en múltiples investigaciones [1, 9, 15], se posiciona como una herramienta robusta y de código abierto capaz de gestionar cargas elevadas. Su arquitectura ligera permite escalar el sistema sin comprometer el rendimiento, lo que es esencial para UrbanTracker, donde múltiples usuarios y administradores acceden simultáneamente a datos de ubicación. Esta robustez asegura que el sistema mantenga la integridad de los datos críticos, como coordenadas GPS, en escenarios de alta demanda urbana.

\subsection{Eficiencia, Recursos Limitados y Escalabilidad}

En UrbanTracker se diferencia al utilizar dispositivos móviles existentes para el envío de coordenadas GPS, los cuales demandan una comunicación que no agote la batería ni los datos, lo cual se demuestra que MQTT fue diseñado precisamente para superar esas limitaciones:

Bajo consumo de recursos:

MQTT se destaca por su eficiencia en el uso de recursos, un aspecto crítico para UrbanTracker dado el empleo de dispositivos móviles como sensores GPS. El protocolo envía únicamente encabezados compactos y payloads ligeros, minimizando el ancho de banda requerido; estudios comparativos indican que MQTT consume hasta un 90% menos datos que HTTP para transmisiones equivalentes. Esta ligereza se traduce en un ahorro significativo de energía, ya que reduce la actividad de la radio del dispositivo móvil, extendiendo la duración de la batería.

Los trabajos relacionados refuerzan esta ventaja. El artículo [1] sobre seguimiento de mascotas muestra que MQTT permite operaciones continuas con un consumo de batería reducido en un 40-50%, comparado con alternativas más pesadas. Aplicaciones en entornos con recursos limitados, como el control de invernaderos [2, 9, 15], acuarios [4] y monitoreo de frigoríficos [6], confirman que MQTT es ideal para dispositivos con capacidades restringidas, manteniendo la funcionalidad sin agotar rápidamente las baterías. En UrbanTracker, esto asegura que el dispositivo móvil del conductor opere durante turnos completos de trabajo sin necesidad de recargas frecuentes, lo que es esencial para la viabilidad económica y operativa del sistema.

Además, el estudio de rastreo GPS de bajo costo con Arduino [12] valida que MQTT no compromete la precisión de las coordenadas, logrando un equilibrio óptimo entre eficiencia y rendimiento. Esta capacidad permite a UrbanTracker utilizar hardware existente sin inversiones adicionales en equipos especializados, alineándose con el objetivo de accesibilidad para entidades de transporte público.

\begin{figure}[h]
\centering
\includegraphics[width=0.5\textwidth]{./graphics/Imagen-6.jpg}
\caption{Imagen correspondiente al artículo [6]}
\end{figure}

Diseño para Redes Inestables:

MQTT incorpora niveles de calidad de servicio (QoS) que lo hacen adecuado para entornos con conectividad variable, como las redes móviles en áreas urbanas. Los niveles QoS 0, 1 y 2 permiten configurar la entrega de mensajes según la criticidad: desde "al menos una vez" hasta "exactamente una vez", asegurando que los datos se transmitan incluso en condiciones de señal débil. En UrbanTracker, esto significa que si un vehículo entra en una zona de baja cobertura, las coordenadas GPS se almacenan temporalmente y se envían al broker Mosquitto tan pronto como la conexión se restablezca, evitando pérdidas de información.

Esta resiliencia es validada en estudios que aplican MQTT en escenarios móviles, donde las interrupciones son comunes. Por ejemplo, el protocolo garantiza la continuidad del flujo de datos sin requerir reconexiones manuales, lo que es esencial para el seguimiento en tiempo real de rutas de transporte público.

Escalabilidad Comprobada:

La arquitectura desacoplada de MQTT facilita la escalabilidad en sistemas de gran envergadura, como UrbanTracker, donde múltiples usuarios y administradores acceden simultáneamente a datos de ubicación. A diferencia de modelos cliente-servidor tradicionales, donde cada conexión consume recursos del servidor, MQTT utiliza un broker central que gestiona suscripciones de manera eficiente, permitiendo que el sistema crezca sin un aumento lineal en la carga.

Estudios como la implementación de arquitecturas distribuidas para gestión de eventos en tiempo real [13] demuestran que MQTT maneja multitudes de eventos con latencias constantes, incluso con miles de suscriptores. El análisis comparativo de protocolos [14] confirma su superioridad sobre alternativas como HTTP en escenarios de alta concurrencia, ya que el broker Mosquitto puede soportar conexiones masivas sin degradación del rendimiento. En UrbanTracker, esto asegura disponibilidad alta, permitiendo que el sistema escale para cubrir flotas enteras de transporte público sin comprometer la eficiencia en la transmisión de coordenadas GPS.

\subsection{Seguridad}

Se exige proteger los datos de acceso y ubicación mediante cifrado. Un estudio específico sobre seguridad de MQTT en IoT [3] confirma que, aunque el protocolo base no incluye cifrado nativo, su integración con TLS/SSL proporciona una capa segura sin afectar significativamente la latencia. En UrbanTracker, esto se traduce en el uso de certificados para autenticar conexiones, previniendo accesos no autorizados y garantizando que las coordenadas GPS no sean interceptadas durante la transmisión.

Además, en el contexto de UrbanTracker, la seguridad es primordial para proteger la privacidad de los usuarios y conductores, evitando posibles ataques de intermediarios que podrían comprometer la integridad de los datos de ubicación y generar riesgos para la seguridad personal.