\section{Marco teórico y trabajos relacionados}

Cuando empecé a investigar sistemas de rastreo en tiempo real, me di cuenta de que básicamente unen dispositivos que capturan posiciones GPS con formas de enviar datos al instante. Esto permite ver ubicaciones en mapas, lanzar alertas o alimentar algoritmos de análisis. Estas soluciones se han vuelto clave para mejorar la movilidad en ciudades, dando una vista clara del estado de una flota y ayudando a predecir retrasos. En transporte público, el boom de IoT ha impulsado el uso de sensores, módulos GPS y plataformas ligeras de mensajería que dan info continua sin necesitar infraestructuras pesadas. Por ejemplo, en \cite{karne2022iot} describen un sistema IoT que usa GPS en buses y MQTT para mandar coordenadas en vivo, demostrando que se puede lograr seguimiento estable y barato incluso en redes con conectividad irregular. El estudio resalta cómo sensores económicos pueden cortar costos operativos en 20-30\% frente a sistemas tradicionales, aunque necesitan calibración inicial para evitar fallos en zonas urbanas densas.

La arquitectura del sistema se vuelve superimportante cuando crece el número de vehículos o necesitas actualizaciones frecuentes. A medida que los datos aumentan, es clave usar enfoques que manejen grandes volúmenes de trayectorias sin perder rendimiento. Según \cite{cervantes2022arquitectura}, una arquitectura distribuida soporta miles de actualizaciones simultáneas con alta tolerancia a fallos, ideal para transporte con escalabilidad progresiva. Pero trae retos como complejidad en sincronización y riesgo de latencias variables en redes distribuidas. Por eso, elegir entre un backend monolítico modular y microservicios depende del contexto. Trabajos como \cite{nugraha2023api} muestran que APIs basadas en microservicios con WebSocket mejoran predicciones de llegada integrando datos externos como tráfico. Sin embargo, pasar a microservicios añade complejidad en despliegue y monitoreo, subiendo costos de mantenimiento en 15-25\% según experiencias. Por eso, UrbanTracker empezó con un monolito modular, manteniendo separación lógica y permitiendo evolución a microservicios sin reescribir todo, equilibrando simplicidad y flexibilidad.

En el lado móvil, los frameworks multiplataforma han pegado fuerte por la rapidez en desarrollo. Herramientas como React Native y Flutter dejan crear apps nativas con un solo código, bajando costos y tiempos. Análisis comparativos dicen que la elección depende de la experiencia del equipo y capacidades de integración. React Native, con JavaScript, destaca por madurez, ecosistema amplio y facilidad para integrar librerías de mapas. Aunque limita acceso a APIs nativas avanzadas vs Flutter. Por estas razones, y para acelerar la app del conductor, UrbanTracker eligió React Native, asegurando compatibilidad y buen rendimiento en Android, común en transporte urbano. Se alinea con tendencias donde React Native manda en apps productivas, mientras Flutter gana en interfaces interactivas.

Para comunicación en tiempo real, UrbanTracker usa MQTT, un protocolo pub/sub ligero. Como explican \cite{villagra2021mqtt}, arquitecturas orientadas a eventos con MQTT son súper escalables y consumen pocos recursos. Comparaciones entre protocolos muestran que aunque WebSocket puede ser más eficiente en CPU y memoria en algunos casos, MQTT generalmente gana en estabilidad y uso de datos cuando dispositivos móviles cambian conectividad. Por ejemplo, en entornos con cobertura intermitente, MQTT mantiene conexiones persistentes con menos overhead, crucial para apps de transporte donde buses pasan por zonas sin señal. Sin embargo, MQTT necesita configuración cuidadosa del broker para evitar cuellos de botella en alta concurrencia, y alternativas como AMQP dan mayor robustez en transacciones críticas, aunque con más complejidad.

Otro punto clave es la seguridad en transmisión. Como detalla \cite{shukla2025oauth}, protocolos OAuth2 y JWT permiten autenticación fuerte en APIs distribuidas, asegurando acceso solo a autorizados. En MQTT, estudios como \cite{aguirre2020mqtt} proponen capas de cifrado para proteger mensajes contra interceptación. UrbanTracker incorpora autenticación JWT en el backend, siguiendo estas prácticas. Aunque JWT tiene limitaciones en revocación inmediata, se ve adecuado para escenarios simples.

Más allá de lo técnico, vale analizar el impacto en el contexto urbano. Sistemas parecidos han bajado congestión optimizando rutas en tiempo real, como en Londres o Singapur. Pero traen desafíos éticos como privacidad de datos de ubicación, solucionables con políticas claras de retención y anonimización. UrbanTracker considera esto diseñando interfaces que minimicen exposición de datos sensibles.

En resumen, el marco teórico respalda decisiones técnicas: usar mapas y geolocalización, elegir React Native para móvil, implementar backend Spring Boot con APIs REST más MQTT para tiempo real y adoptar seguridad JWT/OAuth2. Investigaciones confirman que la combinación es viable y efectiva para sistemas de seguimiento vehicular que operan en entornos urbanos con diferentes niveles de conectividad y escalabilidad. Esta revisión valida el enfoque e identifica oportunidades para mejoras futuras, como IA para predicciones más precisas.
Otro aspecto clave en el diseño de sistemas de rastreo es manejar datos en tiempo real. Trabajos como \cite{fernandez2023bigdata} destacan cómo procesar streams de datos mejora predicción de tiempos de llegada. En este sentido, UrbanTracker se beneficia de arquitecturas que integran bases NoSQL como MongoDB para volúmenes variables de ubicaciones, aunque optamos por PostgreSQL por su robustez en consultas espaciales y consistencia transaccional.

La experiencia de proyectos similares en ciudades europeas, como el de Londres, muestra que integrar APIs de clima y tráfico puede cortar errores de predicción en 20-30\%. UrbanTracker, aunque inicia básico, está diseñado para agregar estas mejoras futuras con módulos plug-in, manteniendo arquitectura modular.

En usabilidad, estudios como \cite{gomez2024ux} enfatizan interfaces adaptativas para usuarios con distintos niveles de alfabetización digital. Nuestras pruebas iniciales confirman que el diseño intuitivo de UrbanTracker facilita adopción, incluso entre mayores, alineándose con tendencias globales de inclusión digital en servicios públicos.

Además, seguridad en comunicaciones IoT ha avanzado mucho. Más allá de MQTT, protocolos como CoAP ofrecen alternativas ligeras, pero MQTT sigue superior en estabilidad para móviles. La revisión de \cite{lopez2022security} valida nuestro enfoque JWT, recomendando cifrado end-to-end para datos sensibles, que implementaremos después.

Por último, el impacto social de estas tecnologías no se puede ignorar. Proyectos en ciudades en desarrollo, como Curitiba en Brasil, prueban que sistemas de info en tiempo real pueden subir uso de transporte público en 25\%, bajando emisiones y congestión. UrbanTracker se ubica ahí, contribuyendo al debate sobre movilidad sostenible y equidad urbana.