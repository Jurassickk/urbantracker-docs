\section{Marco teórico y trabajos relacionados}
Los sistemas de rastreo en tiempo real combinan dispositivos que obtienen posiciones GPS con canales de comunicación instantáneos para mostrar ubicaciones en mapas o generar notificaciones. Por ejemplo, soluciones basadas en Internet de las Cosas (IoT) han utilizado GPS en buses y mensajería ligera para ofrecer información en vivo, resultando escalables y confiables. En \cite{karne2022iot} se propone un sistema IoT donde vehículos equipados con GPS envían coordenadas mediante MQTT, logrando un seguimiento continuo y económico de autobuses.

La arquitectura del sistema es crítica a medida que crecen el número de vehículos y la frecuencia de actualización. Según \cite{cervantes2022arquitectura}, una arquitectura distribuida puede procesar miles de trayectorias en tiempo real con alto rendimiento y tolerancia a fallos. En este sentido, diseñar el backend con un monolito modular (patrón DDD simple) o con microservicios permite escalar componentes de forma independiente. \cite{nugraha2023api} demostró que APIs basadas en microservicios y WebSocket mejoran la estimación de llegada de autobuses al combinar datos en tiempo real con tráfico externo. Sin embargo, la migración a microservicios agrega complejidad operativa; se optó por un monolito modular que facilite futuras migraciones sin grandes refactorizaciones.

En el cliente móvil, los frameworks multiplataforma ofrecen rapidez de desarrollo. Estudios comparativos indican que React Native y Flutter permiten crear apps nativas con una sola base de código, seleccionando según la experiencia del equipo. React Native (JavaScript) se destaca por su amplia comunidad y facilidad de integración con librerías de mapas. En el proyecto UrbanTracker, se seleccionó React Native para acelerar la implementación de la app del conductor.

Para la comunicación en tiempo real se utiliza MQTT, un protocolo de mensajería pub/sub ligero. \cite{villagra2021mqtt} demuestran que arquitecturas orientadas a eventos basadas en MQTT ofrecen bajo consumo de recursos y alta escalabilidad para IoT. Comparaciones entre protocolos muestran que WebSocket suele ser más eficiente en CPU y memoria, mientras que MQTT es más eficiente en uso de red y en velocidad en ciertos escenarios. Esto hace que MQTT sea apropiado cuando los dispositivos (teléfonos móviles) tienen conexiones variables.

La seguridad en la transmisión es otra consideración importante. \cite{shukla2025oauth} describe cómo los protocolos OAuth2 y JWT proporcionan autenticación robusta en APIs distribuidas, asegurando que solo entidades autorizadas accedan a los datos. En el contexto de MQTT, \cite{aguirre2020mqtt} proponen capas de cifrado para proteger los mensajes contra interceptación. En UrbanTracker se adopta autenticación mediante JWT en el backend, siguiendo estas buenas prácticas de seguridad.

En síntesis, el marco teórico respalda las decisiones de diseño: integrar mapas y servicios de geolocalización, emplear React Native en el móvil, usar Spring Boot en el servidor con REST y MQTT para tiempo real, y aplicar mecanismos JWT/OAuth2 para seguridad. Las referencias revisadas confirman que esta combinación tecnológica es viable y efectiva para sistemas de seguimiento vehicular en entornos urbanos.