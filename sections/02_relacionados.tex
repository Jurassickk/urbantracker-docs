\section{Marco teórico y trabajos relacionados}

Antes de comenzar a desarrollar UrbanTracker, tuve que realizar una investigación exhaustiva. No soy un experto que lo sepa todo, por lo que empecé desde cero, buscando en Google de manera intensa, leyendo artículos, blogs y videos de YouTube de personas que claramente entienden más que yo sobre el tema. Al final, lo que entendí es que estos sistemas de seguimiento en tiempo real son básicamente iguales en todas partes: GPS combinado con envío de datos constante. Suena simple, pero cuando se ve en el mapa, con el bus moviéndose y las alertas apareciendo, resulta mágico. Imagina pasar de no saber cuándo llega el bus a saber que faltan tres minutos. La gente puede organizar su día mejor, y eso cambia todo.

Lo que más me sorprendió fue darme cuenta de que el transporte público está lleno de sensores, GPS y toda esa tecnología IoT. Recuerdo un ejemplo que vi, donde combinaban GPS con MQTT para rastrear buses, y funcionaba incluso en zonas con cobertura mala. Me quedé pensando si en serio se puede hacer algo así sin gastar mucho dinero. Resulta que sí, y además se puede ahorrar un 20-30% en costos con sensores baratos. Aunque hay que tener cuidado en ciudades con muchos edificios altos, porque el GPS se vuelve loco. En mi opinión, esto es realmente fascinante porque democratiza el acceso a tecnologías avanzadas. Yo pienso que es una oportunidad para innovar en entornos con recursos limitados.

Pero cuando el sistema crece, ahí vienen los problemas. Cervantes \cite{cervantes2022arquitectura}, que parece saber mucho de esto, explica que una arquitectura distribuida puede manejar miles de actualizaciones simultáneas sin colapsar. Suena bien, sin embargo, trae sus complicaciones, como sincronizar todo, latencias extrañas que tienen despierto por las noches pensando cómo solucionarlo. Toda esa parte técnica que, al final, quita el sueño. Como desarrollador, creo que es importante anticipar estos desafíos desde el inicio. De hecho, la planificación previa ahorra muchos dolores de cabeza.

¿Microservicios o un monolito modular? Esa fue la duda que tuve. Vi casos donde usaban microservicios con WebSocket para integrar datos de tráfico y mejorar predicciones, pero el mantenimiento se vuelve complicado y sube los costos un 15-25%. Al final, opté por un monolito modular para crecer poco a poco. Es como construir una casa: mejor empezar con cimientos sólidos que no intentar todo de golpe. Además, esta elección me permitió mantener el control sobre el proyecto sin complicaciones innecesarias. Por mi parte, prefiero la simplicidad cuando es posible.

\begin{figure}[h]
    \centering
    \includegraphics[width=0.4\textwidth]{graphics/1-img.png}
    \caption{Arquitectura en capas de UrbanTracker mostrando las cuatro capas fundamentales: presentación (interfaces React y React Native), aplicación (servicios de coordinación), dominio (lógica de negocio) e infraestructura (PostgreSQL y MQTT).}
    \label{fig:architecture}
\end{figure}

Para las aplicaciones móviles, entre React Native y Flutter \cite{macias2021flutter}, me quedé con React Native porque ya manejaba JavaScript y hay muchas librerías para mapas. Flutter es más elegante, no lo niego, pero para lo que necesitaba, React Native era perfecto. Además, la mayoría de la gente usa Android, así que encajaba con mi presupuesto limitado. Me parece interesante cómo las herramientas disponibles influyen en las decisiones técnicas. Asimismo, la familiaridad con una tecnología puede acelerar el desarrollo.

MQTT contra WebSocket \cite{werlinder2020mqtt}: MQTT me convenció porque resiste bien cuando las conexiones son inestables. Piensa en un bus que entra en un túnel, pierde señal, sale. ¿Qué pasa entonces? Con MQTT se reconecta automáticamente. WebSocket consume más CPU, pero MQTT es más confiable cuando la señal va y viene. Solo hay que configurar el broker para que no se sobrecargue con muchos dispositivos. Por ejemplo, en entornos con conectividad irregular, MQTT es superior. Es importante destacar que la estabilidad es clave en aplicaciones móviles.

Y la seguridad, ahí me preocupé mucho. OAuth2 y JWT \cite{shukla2025oauth} son lo estándar para sistemas distribuidos, así que los implementé. Para MQTT añadí cifrado, porque no se puede dejar que cualquiera vea las ubicaciones. JWT tiene una debilidad que no me gusta: no se puede revocar inmediatamente si algo parece mal. Pero para el tamaño inicial del proyecto, está bien. Creo que la seguridad debe ser una prioridad desde el principio. Por otro lado, hay que equilibrar seguridad con usabilidad.

Otra cosa que no imaginaba es cómo estos sistemas ayudan con los atascos. Se ha demostrado \cite{kapser2020sustainability} que optimizar rutas en tiempo real reduce el tráfico en varias ciudades. Pero hay que ser cuidadoso con la privacidad. Por eso, en UrbanTracker, diseñé las interfaces para mostrar solo lo esencial, sin datos extra. Esto es crucial para ganar la confianza de los usuarios. En realidad, la privacidad es un tema que me apasiona.

Después de toda esta investigación, confirmé que mis decisiones tenían sentido: geolocalización, React Native para móvil, Spring Boot en el backend, MQTT para tiempo real y JWT para seguridad. Los artículos indican que esta combinación funciona en ciudades con conectividad irregular. Y está claro que después se puede integrar IA para predicciones; eso ya lo tengo en mente para el futuro. Me entusiasma pensar en las posibilidades de expansión. Desde mi perspectiva, el futuro de estos sistemas es prometedor.

Sobre el streaming de datos \cite{nugraha2023api}, leí artículos que explican por qué procesar información al instante mejora las estimaciones. MongoDB podría manejar muchas ubicaciones, pero al final elegí PostgreSQL porque hace consultas espaciales mejor y es más consistente. En mi experiencia, la elección de la base de datos es fundamental para el rendimiento. De hecho, PostgreSQL ha sido una elección sólida para mis proyectos anteriores.

Se ha reducido \cite{juric2021predictivo} errores de predicción un 20-30% incorporando datos de clima y tráfico en algunos estudios. Mi arquitectura modular me permite añadir esos módulos después sin problemas. Para la usabilidad, me di cuenta de que las interfaces deben adaptarse a diferentes niveles de conocimientos digitales. Lo probé con mis abuelos, que ahora usan smartphones, y funciona bien. Lo cual es genial, porque democratiza el acceso. Realmente, la inclusión es un aspecto que valoro mucho. Yo pienso que la accesibilidad es esencial para el éxito.

En seguridad IoT, MQTT sigue siendo lo mejor para móviles, aunque CoAP es más ligero. López \cite{villagra2021mqtt} confirma que JWT funciona bien en sistemas distribuidos y recomienda cifrado end-to-end más adelante. Sin embargo, para el alcance actual, JWT es adecuado. Me parece que las recomendaciones de expertos son valiosas para guiar las decisiones.

Y el impacto social en algunas ciudades \cite{karne2022iot}, la información en tiempo real ha aumentado el uso del transporte público y reducido emisiones. UrbanTracker puede formar parte de esa conversación sobre sostenibilidad y movilidad justa. Al final, eso es lo que me motivó a hacerlo: no solo para aprobar algo, sino porque siento que puede marcar una diferencia real en la vida cotidiana. Como colombiano, creo que proyectos como este pueden mejorar la calidad de vida en lugares como Neiva. En realidad, es gratificante contribuir a la comunidad local.

\begin{figure}[h]
    \centering
    \includegraphics[width=0.4\textwidth]{graphics/2-img.png}
    \caption{Cobertura geográfica del sistema UrbanTracker en ciudades como Neiva, mostrando las principales rutas de transporte público y áreas de operación del proyecto.}
    \label{fig:neiva-coverage}
\end{figure}
