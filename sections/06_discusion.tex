\section{Discusión}

UrbanTracker muestra un buen rendimiento en escenarios pequeños y medianos, y nuestros experimentos confirman que las arquitecturas con MQTT funcionan bien para rastreo vehicular con actualizaciones frecuentes y comunicación confiable. La mezcla de tecnologías nos permitió crear una solución flexible, adaptable a diferentes necesidades operativas y entornos con conectividad variable.

\begin{figure}[h]
\centering
\includegraphics[width=0.85\textwidth]{graphics/7-img.png}
\caption{Análisis comparativo de UrbanTracker vs sistemas tradicionales de rastreo vehicular}
\label{fig:analisis-comparativo}
\end{figure} Aunque los resultados nos animan, también revelan áreas que necesitamos mejorar para fortalecer el sistema. Por ejemplo, completar más pruebas unitarias e integración ayudaría a detectar fallos temprano y aumentar la estabilidad antes de llevarlo a producciones más exigentes. Además, añadir análisis predictivo, como modelos para estimar tiempos de llegada o alertas basadas en historial, podría subir mucho el valor funcional.

La arquitectura modular que usamos resultó ser una buena idea, ya que da una base sólida para migrar a microservicios si el sistema crece. Esta modularidad permite escalar cada componente sin afectar al resto, abriendo puertas a evoluciones graduales según necesidades reales. Además, la experiencia nos sugiere que podríamos adaptar la solución a otros contextos urbanos similares, como transporte escolar o flotas corporativas.

\subsection{Implicaciones sociales y económicas}

Desde lo social, UrbanTracker ayuda a reducir la ansiedad de los usuarios con info precisa, fomentando más uso del transporte público y menos dependencia de carros privados. Esto trae beneficios como menos congestión y mejor calidad de vida. Económicamente, permite a operadores optimizar rutas y recursos, reduciendo costos operativos en 20-30\% con mejor planificación. Pero la implementación inicial requiere inversión en móviles y capacitación, lo que puede ser un obstáculo para municipios con presupuestos limitados.

\subsection{Implicaciones ambientales}

Al promover uso eficiente del transporte público, UrbanTracker contribuye a bajar emisiones de CO2. Estudios dicen que sistemas de geolocalización reducen tráfico vehicular en ciudades al mejorar predictibilidad, alineándose con objetivos de sostenibilidad global.

\subsection{Limitaciones técnicas}

Dicho esto, reconocemos que escalabilidad y robustez necesitan evaluación más profunda en escenarios con más vehículos, actualizaciones frecuentes o bases de usuarios grandes. Aunque MQTT funcionó estable en pruebas iniciales, podría variar en alta demanda, así que recomendamos pruebas de estrés para ver límites. Por otro lado, seguridad y privacidad de datos de ubicación requieren análisis constante, ya que mal manejo podría arriesgar integridad o info de usuarios. Necesitamos reforzar autenticación, cifrado y gestión de historial.

En conjunto, UrbanTracker es una base sólida para herramienta moderna de rastreo, pero necesita iteraciones para validación empírica, optimización y mejora de seguridad. Esto lo hará más robusto para complejidades urbanas.

Además, limitaciones como dependencia de GPS y precisión en áreas densas (edificios altos) podrían afectar fiabilidad. La batería de móviles también limita actualizaciones continuas.

A mayor escala, plantea preguntas sobre equidad en acceso a tecnologías de movilidad. Beneficia a quienes tienen smartphones modernos, pero excluye a otros sin acceso. Resalta necesidad de políticas de inclusión digital en expansiones futuras.

La sostenibilidad ambiental es crucial. Aunque promueve transporte público, reduciendo CO2, el sistema consume energía en backend y procesamiento. Un análisis de ciclo de vida cuantificaría si beneficios superan costos. Tecnologías como edge computing ayudarían a distribuir procesamiento y bajar huella de carbono.

En escalabilidad internacional, enfrenta desafíos regulatorios. Países tienen normativas variadas sobre privacidad de ubicación, requiriendo adaptaciones. Por ejemplo, en UE, GDPR implica anonimización extra y consentimiento. Sugiere modularidad regulatoria para configuraciones personalizadas.

Finalmente, resalta necesidad de enfoque holístico en desarrollo de sistemas inteligentes. UrbanTracker no es solo técnico, sino catalizador de cambios sociales y urbanos. Su éxito depende de integrar éticas, ambientales y sociales desde diseño, asegurando innovación contribuya a movilidad sostenible e inclusiva.
Desde una perspectiva técnica, UrbanTracker valida la efectividad de combinar protocolos IoT con frameworks web modernos. La estabilidad de MQTT en entornos urbanos variables confirma hallazgos previos, pero también revela la necesidad de optimizaciones para escalas mayores. Por ejemplo, la implementación de clustering en el broker podría mitigar cuellos de botella en picos de demanda, como durante eventos masivos o horas pico.

Otro punto de discusión es la sostenibilidad a largo plazo. Aunque el sistema reduce costos operativos, el mantenimiento de dispositivos móviles y la actualización de software representan desafíos continuos. Estudios comparativos con sistemas tradicionales sugieren que, después de 2-3 años, los ahorros superan las inversiones iniciales, pero esto depende de la adopción masiva y el soporte institucional.

En términos de equidad social, surge la pregunta: ¿quién se beneficia más? Nuestros datos indican que usuarios con acceso a smartphones modernos ven mejoras significativas, pero aquellos sin dispositivos quedan excluidos. Esto plantea la necesidad de estrategias complementarias, como quioscos públicos o integraciones con apps de mensajería popular como WhatsApp, para extender el alcance.

Ambientalmente, el impacto positivo es claro, pero cuantificarlo requiere métricas precisas. Estimaciones preliminares sugieren una reducción del 10-15\% en viajes en vehículo privado, pero factores como el crecimiento urbano podrían diluir estos beneficios. La integración con sistemas de ciudades inteligentes, como sensores de calidad del aire, podría amplificar el efecto positivo.

Desde lo económico, UrbanTracker no solo optimiza rutas, sino que genera datos valiosos para planificación urbana. Administradores pueden usar insights para ajustar frecuencias de servicio, reduciendo sobrecargas en rutas populares. Sin embargo, la monetización de estos datos debe balancearse con privacidad, evitando modelos que prioricen ganancias sobre beneficios públicos.

Finalmente, el proyecto destaca la importancia de la colaboración interdisciplinaria. Ingenieros, urbanistas y sociólogos deben trabajar juntos para asegurar que soluciones técnicas aborden problemas humanos reales. UrbanTracker es un ejemplo de cómo la tecnología puede catalizar cambios sociales, pero su éxito depende de implementación responsable y evaluación continua.