\section{Discusión}

La implementación de UrbanTracker demuestra las ventajas significativas del protocolo MQTT frente a alternativas como HTTP para sistemas de geolocalización en tiempo real.

\subsection{MQTT vs HTTP para Comunicación en Tiempo Real}

Los resultados obtenidos confirman que MQTT habilita flujos ligeros y de baja latencia. Frente al modelo Request/Response de HTTP, el enfoque publicación/suscripción reduce la sobrecarga y escala mejor con múltiples vehículos y usuarios. Esto es particularmente importante en el contexto de transporte público donde se requieren actualizaciones constantes de posición con el mínimo consumo de recursos.

\begin{figure}[h]
\centering
\includegraphics[width=0.5\textwidth]{./graphics/Imagen-14.jpg}
\caption{Imagen correspondiente al artículo [14]}
\end{figure}

\subsection{Ventajas de la Arquitectura Propuesta}

La arquitectura distribuida de UrbanTracker presenta varios beneficios clave:

\begin{figure}[h]
\centering
\includegraphics[width=0.5\textwidth]{./graphics/Imagen-11.jpg}
\caption{Imagen correspondiente al artículo [11]}
\end{figure}

\begin{itemize}
\item \textbf{Eficiencia en el uso de recursos}: El protocolo MQTT es significativamente más ligero que HTTP para el manejo de eventos frecuentes
\item \textbf{Baja latencia}: Las actualizaciones de geolocalización llegan prácticamente en tiempo real a los usuarios
\item \textbf{Escalabilidad}: El modelo pub/sub permite manejar fácilmente el crecimiento del número de vehículos y usuarios
\item \textbf{Costo-efectividad}: El uso de dispositivos móviles como sensores elimina la necesidad de hardware especializado costoso
\end{itemize}

\subsection{Desafíos y Consideraciones}

Si bien la implementación demuestra éxito, se identifican áreas que requieren atención:

\begin{itemize}
\item \textbf{Seguridad}: Es necesario reforzar las medidas de cifrado y autenticación para proteger los datos de ubicación
\item \textbf{Gestión de dispositivos}: Se requieren políticas claras de uso y reconexión para los dispositivos móviles de los conductores
\item \textbf{Administración}: Los administradores necesitan herramientas adicionales para el monitoreo del sistema completo
\end{itemize}

Además, la comparación con otros protocolos resalta la superioridad de MQTT en entornos de IoT urbanos, donde la conectividad intermitente y los recursos limitados son comunes.