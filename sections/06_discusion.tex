\section{Discusión}
El sistema UrbanTracker demuestra un rendimiento adecuado en escenarios de pequeña y mediana escala, y los experimentos realizados confirman que las arquitecturas soportadas en MQTT resultan viables para aplicaciones de rastreo vehicular que dependen de actualizaciones frecuentes y comunicación fiable. La combinación de tecnologías implementadas permitió construir una solución flexible, capaz de adaptarse a distintas necesidades operativas y a entornos con conectividad variable. Aunque los resultados son alentadores, también revelan que aún existen áreas en las que es necesario trabajar para fortalecer el sistema. Entre ellas, destaca la necesidad de completar un conjunto más amplio de pruebas unitarias y de integración, ya que estas permitirían identificar fallas tempranas y mejorar la estabilidad general antes de llevar el sistema a producciones más exigentes. Del mismo modo, la incorporación de análisis predictivo —como modelos para estimar tiempos de llegada o alertas basadas en comportamiento histórico— podría elevar considerablemente el valor funcional del sistema.

La arquitectura modular empleada durante el desarrollo ha demostrado ser un acierto, ya que brinda una base sólida para una futura migración hacia microservicios si el sistema crece y requiere mayor capacidad de procesamiento. Esta modularidad permite que cada componente pueda escalarse o sustituirse sin afectar el resto del sistema, lo que abre la puerta a una evolución gradual basada en las necesidades reales del entorno. Además, la experiencia adquirida durante el diseño e implementación sugiere que la solución podría adaptarse a otros contextos urbanos con características similares, como sistemas de transporte escolar, flotas corporativas o rutas urbanas en municipios con menos recursos tecnológicos.
\subsection{Implicaciones sociales y económicas}
Desde una perspectiva social, UrbanTracker contribuye a reducir la ansiedad de los usuarios al proporcionar información precisa, lo que puede fomentar un mayor uso del transporte público y disminuir la dependencia de vehículos privados. Esto tiene implicaciones positivas en la reducción de congestión urbana y mejora de la calidad de vida. Económicamente, el sistema permite a los operadores optimizar rutas y recursos, potencialmente reduciendo costos operativos en un 20-30\% mediante una mejor planificación. Sin embargo, la implementación inicial requiere inversión en dispositivos móviles y capacitación, lo que podría ser un obstáculo para municipios con presupuestos limitados.

\subsection{Implicaciones ambientales}
Al promover el uso eficiente del transporte público, UrbanTracker puede contribuir a la disminución de emisiones de CO2. Estudios indican que sistemas de geolocalización pueden reducir el tráfico vehicular en áreas urbanas al mejorar la predictibilidad del servicio, lo que se alinea con objetivos de sostenibilidad global.

\subsection{Limitaciones técnicas}

No obstante, es importante reconocer que la escalabilidad y la robustez del sistema deberán ser evaluadas con mayor profundidad cuando se enfrente a escenarios con un número más elevado de vehículos, una frecuencia mayor de actualizaciones o una base de usuarios considerablemente más amplia. Aunque MQTT mostró un rendimiento estable en pruebas iniciales, su comportamiento podría variar en entornos de alta demanda o con múltiples suscriptores simultáneos, por lo que sería recomendable realizar pruebas de estrés para determinar los límites de la infraestructura. Por otro lado, la seguridad y la privacidad de los datos —especialmente aquellos relacionados con la ubicación en tiempo real— requieren un análisis constante, ya que su mal manejo podría poner en riesgo la integridad del sistema o la información de los usuarios. En este sentido, resulta necesario continuar reforzando los mecanismos de autenticación, cifrado y gestión del historial de datos.

En conjunto, la discusión de los resultados indica que UrbanTracker constituye una base sólida para una herramienta de rastreo vehicular moderna y adaptable, pero que aún necesita iteraciones adicionales orientadas a la validación empírica, la optimización del rendimiento y la mejora continua de la seguridad. Estas acciones permitirán que el sistema evolucione hacia un servicio más robusto y preparado para operar en escenarios urbanos de mayor complejidad.
Además, limitaciones como la dependencia de conectividad GPS y la precisión en áreas urbanas densas (debido a edificios altos) podrían afectar la fiabilidad en ciertos contextos. La batería de los dispositivos móviles también representa un factor limitante para actualizaciones continuas.
Además, limitaciones como la dependencia de conectividad GPS y la precisión en áreas urbanas densas (debido a edificios altos) podrían afectar la fiabilidad en ciertos contextos. La batería de los dispositivos móviles también representa un factor limitante para actualizaciones continuas.

En un nivel más amplio, la implementación de UrbanTracker plantea preguntas sobre la equidad en el acceso a tecnologías de movilidad. Mientras que beneficia a usuarios con smartphones modernos, podría excluir a segmentos de la población con dispositivos obsoletos o sin acceso a datos móviles. Esto resalta la necesidad de considerar políticas de inclusión digital en futuras expansiones.

Finalmente, la discusión de los resultados indica que UrbanTracker constituye una base sólida para una herramienta de rastreo vehicular moderna y adaptable, pero que aún necesita iteraciones adicionales orientadas a la validación empírica, la optimización del rendimiento y la mejora continua de la seguridad. Estas acciones permitirán que el sistema evolucione hacia un servicio más robusto y preparado para operar en escenarios urbanos de mayor complejidad.
Además, limitaciones como la dependencia de conectividad GPS y la precisión en áreas urbanas densas (debido a edificios altos) podrían afectar la fiabilidad en ciertos contextos. La batería de los dispositivos móviles también representa un factor limitante para actualizaciones continuas, especialmente en rutas largas donde el consumo energético es alto.

En un nivel más amplio, la implementación de UrbanTracker plantea preguntas sobre la equidad en el acceso a tecnologías de movilidad. Mientras que beneficia a usuarios con smartphones modernos, podría excluir a segmentos de la población con dispositivos obsoletos o sin acceso a datos móviles. Esto resalta la necesidad de considerar políticas de inclusión digital en futuras expansiones, como el desarrollo de aplicaciones ligeras o interfaces alternativas para dispositivos de gama baja.

La sostenibilidad ambiental es otro aspecto crucial. Aunque UrbanTracker promueve el uso del transporte público al reducir incertidumbres, su impacto neto depende de cómo se integre con estrategias más amplias de reducción de emisiones. Por ejemplo, si los usuarios optan por el transporte público en lugar de vehículos privados, el beneficio es claro; sin embargo, si el sistema incentiva viajes innecesarios, podría tener efectos contraproducentes.

Finalmente, la discusión de los resultados indica que UrbanTracker constituye una base sólida para una herramienta de rastreo vehicular moderna y adaptable, pero que aún necesita iteraciones adicionales orientadas a la validación empírica, la optimización del rendimiento y la mejora continua de la seguridad. Estas acciones permitirán que el sistema evolucione hacia un servicio más robusto y preparado para operar en escenarios urbanos de mayor complejidad, contribuyendo así a la transformación digital del transporte público.
Además, la implementación de UrbanTracker plantea preguntas sobre la equidad en el acceso a tecnologías de transporte inteligente. En ciudades con brechas digitales significativas, como muchas en América Latina, no todos los usuarios tienen acceso a smartphones modernos o conexiones estables. Esto podría exacerbar desigualdades, donde solo ciertos grupos socioeconómicos se beneficien de la información en tiempo real. Futuras versiones deberían considerar interfaces alternativas, como aplicaciones web optimizadas para dispositivos de gama baja o integraciones con sistemas de mensajería masiva como SMS.

Otro aspecto crítico es el impacto ambiental. Aunque UrbanTracker promueve el uso del transporte público, reduciendo potencialmente las emisiones de CO2, la operación del sistema mismo consume energía, especialmente en el backend y el procesamiento de datos. Un análisis de ciclo de vida completo sería necesario para cuantificar si los beneficios ambientales superan los costos energéticos. Tecnologías como el edge computing podrían ayudar a distribuir el procesamiento y reducir la huella de carbono del sistema.

En términos de escalabilidad internacional, UrbanTracker enfrenta desafíos regulatorios. Diferentes países tienen normativas variadas sobre privacidad de datos de ubicación, lo que requiere adaptaciones específicas. Por ejemplo, en la Unión Europea, el cumplimiento con GDPR implica medidas adicionales de anonimización y consentimiento explícito. Esto sugiere que el sistema debe ser diseñado con modularidad regulatoria, permitiendo configuraciones personalizadas según jurisdicciones.

Finalmente, la discusión resalta la necesidad de un enfoque holístico en el desarrollo de sistemas de transporte inteligente. UrbanTracker no es solo una herramienta técnica, sino un catalizador para cambios sociales y urbanos más amplios. Su éxito dependerá de cómo integre consideraciones éticas, ambientales y sociales desde las primeras etapas de diseño, asegurando que la innovación tecnológica contribuya efectivamente a una movilidad urbana más sostenible e inclusiva.