\section{Discusión}
El sistema UrbanTracker demuestra un rendimiento adecuado a pequeña escala y valida en la práctica la viabilidad de arquitecturas basadas en MQTT para rastreo vehicular. La integración de tecnologías modernas permitió una solución eficiente y flexible, aunque aún quedan retos por abordar, como la implementación de pruebas unitarias completas y la incorporación de análisis predictivo.

La arquitectura modular facilita la futura migración a microservicios, y la experiencia obtenida puede ser generalizable a otros contextos urbanos con necesidades similares. Sin embargo, la escalabilidad y robustez deberán evaluarse en escenarios de mayor volumen y diversidad de usuarios. Se recomienda continuar con la validación empírica y la mejora de la seguridad y privacidad de los datos.