\section{Trabajo futuro y extensiones}

UrbanTracker funciona pero en realidad es algo muy básico. Sé que hay muchísimo más que podría hacer si tuviera tiempo, recursos, o si simplemente fuera más inteligente.

Lo que más me atrae es meter machine learning para predecir cuándo llegan los buses. No como estoy haciendo ahora que es básicamente "si salió hace 10 minutos y está a 5 km de distancia, llega en X minutos". Eso es matemática de primaria. Lo que quiero es usar históricos - si llueve, si es viernes, si hay un evento en la ciudad - todo eso afecta cuándo llega el bus. Con Random Forest o redes neuronales podría mejorar la precisión talvez 30-40\%. O talvez menos. No sé. Lo difícil sería que no quiero una caja negra donde el modelo dice "bus llega en 15 minutos" pero no sé por qué. Eso sería peligroso. Necesitaría alguien que sepa de machine learning de verdad, no yo viendo tutoriales en YouTube.

Otra cosa que sueño con hacer es multimodalidad. Que UrbanTracker no sea solo buses sino que integre bicicletas compartidas, scooters, metro, todo. Una persona dice "quiero ir del norte al sur" y la app le muestra: toma este bus 5 cuadras, después un scooter, después a pie. Rutas intermodales optimizadas. Eso sería increíble. Pero también es mucho trabajo. Tendría que hablar con todas esas empresas, conseguir sus APIs, integrarlas. Es complicado.

Sostenibilidad. Me gustaría que la app muestre "si tomas este bus en lugar de tu carro, ahorras X kg de CO2". Que la gente vea en números lo que está ayudando. Reportes mensuales: "tomaste bus 40 veces, evitaste 500 kg de emisiones". Eso podría motivar a la gente. Pero también sería marketing, y no sé si la gente le cree a eso. Probablemente necesitaría colaborar con ONGs ambientales para que verifiquen los números.

Técnicamente, migrar a microservicios sería más limpio. Ahora tengo todo en un monolito. Si el servidor de GPS se cae, cae todo. Si me llegan 10 mil usuarios simultáneos probablemente explota. Con microservicios, cada cosa escalable de forma independiente. Kubernetes para la orquestación, todo en la nube. Pero eso requiere mucho más infraestructura y plata. Y honestamente, para Neiva probablemente es overkill. Pero está ahí como idea.

Privacidad es algo que me preocupa para el futuro. En v1 metí cifrado y anonimización básica. Pero si UrbanTracker crece, un gobierno malo podría querer acceso a todos los datos. "Dónde estuvo cada persona a cada hora". Eso es vigilancia. Necesitaría técnicas más avanzadas - anonimato diferencial, contratos inteligentes, políticas claras sobre qué se guarda y por cuánto tiempo. Eso es complejo legalmente y técnicamente.

Una extensión que sería cool es integración con ciudades inteligentes. APIs de semáforos que me digan "van a estar en rojo los próximos 3 minutos", información de eventos "hay concierto en la plaza mañana a las 8, habrá más gente", datos de clima. Todo eso entraría en las predicciones. Pero ciudades inteligentes reales no existen casi en Colombia. Neiva tiene un semáforo de verdad? No. Entonces es más una idea para el futuro.

Lo que honestamente me encantaría pero sé que es un sueño es que UrbanTracker se vuelva open source de verdad y que otros desarrolladores lo usen y mejoren. Que alguien en Bogotá diga "voy a usar UrbanTracker para Bogotá". Que alguien en El Salvador lo adapte. Que tenga comunidad alrededor. Eso sería épico. Pero también significa dejar que otros toquen tu código, lo critiquen, lo mejoren. Eso es difícil. Mi ego dice que debo ser perfeccionista, pero la realidad es que un proyecto comunitario es mucho más fuerte que uno personal.

Hay un montón de cosas pequeñas también. Agregar notificaciones push - "tu bus está llegando". Audio para usuarios ciegos - eso debería haber hecho en v1. Soporte para múltiples idiomas. Integración con calendario - "detéctame que hoy no tengo que ir a la universidad, no me avises de buses". Cosas que harían la app mejor para cada usuario específico.

Pero honestamente? No sé si haré todo eso. Depende de si alguien financia esto, si consigo colaboradores, si me gradúo y tengo tiempo. Ahora estoy en la universidad, UrbanTracker es un proyecto más en una lista larga. Si termina siendo solo esto que hice - una app que funciona en Neiva y ayuda a mi abuela - estaré contento. No es lo máximo que podría ser, pero es algo real que existe y funciona.

La verdad es que el futuro es incierto. Capaz en 5 años UrbanTracker es esto mismo y nada más. Capaz es usado en 100 ciudades. Capaz alguien se lo roba y lo vuelve millonario. No lo sé. Lo que sé es que hice algo que funciona y que ayuda a la gente en Neiva. Eso es suficiente por ahora. El resto, bueno, es futuro.a
