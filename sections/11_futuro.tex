\section{Trabajo futuro y extensiones}
UrbanTracker ofrece una base sólida para futuras expansiones que pueden enriquecer significativamente su funcionalidad y aplicabilidad. Una línea de desarrollo prioritaria es la integración de algoritmos de inteligencia artificial para predicciones más precisas. Por ejemplo, utilizando modelos de machine learning, el sistema podría anticipar tiempos de llegada basados en datos históricos de tráfico, clima y patrones de uso. Esto no solo mejoraría la experiencia del usuario al proporcionar estimaciones más confiables, sino que también permitiría optimizar rutas en tiempo real, reduciendo congestiones y emisiones.

Otra extensión importante es la incorporación de análisis de big data para insights operativos. Al procesar grandes volúmenes de datos de ubicación, UrbanTracker podría generar reportes automáticos sobre patrones de demanda, eficiencia de rutas y comportamiento de conductores. Esto facilitaría decisiones estratégicas para administradores, como la redistribución de flotas o la planificación de mantenimiento preventivo. Tecnologías como Apache Kafka para procesamiento de streams y Elasticsearch para búsquedas avanzadas podrían integrarse para manejar esta escalabilidad.

En el ámbito de la sostenibilidad, futuras versiones podrían incluir métricas ambientales, como cálculo de huella de carbono por viaje o recomendaciones para modos de transporte alternativos. Integraciones con APIs de ciudades inteligentes permitirían acceso a datos de semáforos, obras viales o eventos locales, mejorando la precisión de las predicciones.

Desde la perspectiva técnica, la migración completa a microservicios permitiría una mayor resiliencia y facilidad de despliegue. Cada módulo —autenticación, geolocalización, mapas— podría escalarse independientemente, soportando cargas variables. Además, la adopción de Kubernetes para orquestación de contenedores facilitaría despliegues en la nube, reduciendo costos operativos.

La privacidad y seguridad también requieren atención continua. Implementaciones futuras podrían incluir técnicas de anonimato diferencial para datos de ubicación, asegurando cumplimiento con regulaciones como GDPR o leyes locales. Autenticación multifactor y cifrado end-to-end fortalecerían la confianza de los usuarios.

Finalmente, la expansión internacional del sistema plantea desafíos culturales y regulatorios. Adaptaciones para diferentes idiomas, monedas y normativas de transporte serían necesarias. Colaboraciones con universidades y empresas podrían acelerar estas extensiones, convirtiendo UrbanTracker en una plataforma global para movilidad urbana inteligente.

Estas propuestas no solo amplían el alcance del proyecto, sino que también lo posicionan como una herramienta innovadora en el ecosistema de transporte público, preparada para evolucionar con las necesidades futuras de las ciudades.