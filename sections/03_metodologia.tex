\section{Metodología}

La metodología de este análisis se basó en una revisión bibliográfica de los 15 trabajos de caso [1-15] para validar la decisión arquitectónica en el contexto de UrbanTracker. El proceso incluyó:

Identificación de Requisitos Críticos: Se definieron los requisitos de tiempo real, envío de ubicación desde el móvil y seguridad como dependientes del protocolo de comunicación.

Mapeo de Evidencia: Se compararon los beneficios probados de MQTT (ligereza, publicación/suscripción, bajo consumo) con estos requisitos, encontrando una correlación directa y consistente en todos los estudios.

La implementación de UrbanTracker está diseñada como una arquitectura distribuida que conecta la aplicación móvil del conductor (el Publicador), un Broker MQTT (como Mosquitto) en la nube, y la plataforma web/móvil del usuario/administrador (el Suscriptor). El protocolo MQTT gestiona el flujo de datos GPS en tiempo real a través de tópicos específicos, asegurando que la información llegue de manera inmediata a la interfaz gráfica.

\begin{figure}[h]
\centering
\includegraphics[width=0.5\textwidth]{./graphics/Imagen-13.jpg}
\caption{Imagen correspondiente al artículo [13]}
\end{figure}

Esta metodología asegura una evaluación rigurosa y basada en evidencia, permitiendo una comparación objetiva entre diferentes protocolos de comunicación IoT y su aplicabilidad en sistemas de transporte urbano.