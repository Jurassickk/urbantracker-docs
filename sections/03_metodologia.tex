\section{Metodología de investigación aplicada}
Arquitectura: (1) Adquisición de datos: la app del conductor obtiene coordenadas con el GPS del dispositivo móvil y publica JSON al bróker; (2) Backend: bróker MQTT (Mosquitto), almacenamiento y API de apoyo; (3) Presentación: apps web con mapas interactivos.

Comunicación: tópicos del tipo /ruta/vehículo que permiten separar productores y consumidores. Para la web se habilita MQTT sobre WebSocket.

Seguridad: autenticación por roles, JWT.

MARCO DE REFERENCIA — Diseño IoT para rastreo vehicular: Según «DESARROLLO DE UN SISTEMA DE LOCALIZACIÓN BASADO EN GPS E IOT: UN ESTUDIO DE CASO EN QUITO» (ISTCT, 2025), un dispositivo con GPS y conectividad GSM/4G envía datos a la nube para visualización en tiempo real.

MARCO DE REFERENCIA — Uso de smartphone como nodo IoT: Según «SISTEMA DE GEOLOCALIZACION DE VEHICULOS RECOLECTORES DE BASURA APLICANDO INTERNET DE LAS COSAS» (UPEA, 2025), es viable sustituir hardware dedicado por dispositivos móviles con GPS para reducir costos de implementación.