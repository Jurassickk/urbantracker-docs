\section{Metodología de investigación aplicada}
El desarrollo de UrbanTracker siguió un enfoque iterativo incremental. Se definió inicialmente una arquitectura de monolito modular en Spring Boot (DDD simple), lo que facilita dividir el sistema en módulos (autenticación, gestión de rutas, datos de ubicación, etc.) sin sacrificar la posibilidad de migrar a microservicios en el futuro. El frontend web se basó en React/Next.js, mientras que la app móvil se construyó con React Native para Android.

Los primeros pasos incluyeron la implementación del módulo de autenticación (Spring Security + JWT) y las APIs REST para usuarios, rutas, conductores y vehículos. Paralelamente, se habilitó la funcionalidad de geolocalización en la app móvil: esta obtiene el GPS del teléfono y publica coordenadas cada pocos segundos en un broker MQTT (Mosquitto). Si el vehículo asignado no cuenta con un dispositivo GPS propio, el sistema recurre al GPS del móvil. El backend Spring Boot se suscribe a los tópicos MQTT definidos (por ruta) y recibe las actualizaciones de posición. En cada mensaje se validan y almacenan las coordenadas en PostgreSQL (diferentes esquemas por módulo, siguiendo DDD), y se pone disponible la última posición en servicios REST para las interfaces web.

Las interfaces de usuario incluyen un visor de mapas (usando Mapbox) que consulta al backend la ubicación más reciente de cada vehículo en servicio. Para la gestión administrativa, se desarrolló un panel web donde el administrador crea y asigna rutas, conductores y vehículos, y puede observar el estado de la flota. Todo el sistema puede ejecutarse en local mediante Docker Compose (servidor PostgreSQL, broker MQTT), estandarizando las variables de entorno.

Durante el proceso se realizaron pruebas unitarias parciales en servicios clave, así como pruebas de integración básicas (E2E) usando datos simulados. La prioridad de desarrollo siguió las especificaciones funcionales definidas (por ejemplo, mostrar rutas y vehículos en mapa). El soporte para sensores externos o plataformas de ciudades inteligentes no fue necesario, ya que el sistema opera de manera autónoma en la infraestructura móvil disponible.