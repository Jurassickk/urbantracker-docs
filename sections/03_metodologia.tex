\section{Metodología de investigación aplicada}

Cuando empecé con esto no sabía bien qué hacer. Iba a mi primera clase de investigación, sacaba lo que me daban de metodologías y... nada me encajaba. Todos hablaban de marcos teóricos y protocolos como si fueran recetas. Yo solo quería que mi código funcionara. Así que básicamente inventé mi propia cosa. No fue Scrum, no fue metodología clásica. Fue más como: construye algo, pruébalo, arreglalo, repite.

Lo primero fue ir a hablar con conductores en Neiva. Literalmente me fui a varias paradas de buses con un cuadernito. Hablé con conductores de ruta, pasajeros que esperaban. Una conductora, que se llama Yolanda, me contó que ella madrugaba todos los días y ni siquiera sabía en qué orden pasaban los buses de su ruta. A veces se le pasaba uno y se perdía toda la mañana. Eso fue lo que más me golpeó. No era una incómoda, era tiempo real de la vida de la gente que se perdía. También fui a hablar con los adminsitradores de una empresa pequeña de transportes y ellos me dijeron que la mayoría de conductores no les pasaban reportes. Trabajaban a ciegas.

Después empecé a leer artículos. Buscaba en Google Scholar, ResearchGate, cualquier lado donde alguien hablara de rastreo de buses. Leí cosas de sistemas en Hong Kong, en Colombia, en Argentina. Algunos funcionaban, otros eran caros, otros estaban desactualizados. Lo que me sorprendió fue que la mayoría usaban tecnologías viejas. No porque las nuevas no existieran, sino porque nadie las había puesto junto.

Hice pruebas en la calle. Cogí un Samsung J2 que mi hermano no usaba, me fui caminando por diferentes zonas de Neiva. En el centro, el GPS se desviaba 50 metros fácilmente por los edificios. En zonas abiertas funcionaba bien. Eso me hizo pensar: no puedo hacer un sistema que dependa de que el GPS sea perfecto. Tiene que tolerar errores.

Para el desarrollo, hice sprints de dos semanas. Primera semana: solo un punto en un mapa. No login, nada. Solo GPS a mapa. Mi cuarto literalmente se convirtió en una central de monitoreo con cables por todos lados. La segunda semana: agregué rutas. La tercera: seguridad básica. Así de simple. Cada semana mostraba algo que funcionaba. Eso me daba motivación para seguir.

Usé tecnologías que ya conocía o que eran fáciles de aprender. Spring Boot porque ya había hecho cosillas con Java. React Native porque sabía JavaScript. PostgreSQL porque tuve que aprenderlo y no fue tan difícil. MQTT porque en un artículo leí que funcionaba sin conexión buena - y era verdad, lo probé múltiples veces. Con 1 barra de señal, MQTT seguía enviando. WebSocket se moría.

Las pruebas vinieron después. No hice pruebas antes del código - eso es bonito en la teoría pero cuando estás solo y aprendiendo es casi imposible. Hice pruebas después, a veces funcionaban, a veces encontraba bugs. Cuando encontraba algo roto, lo arreglaba y probaba de nuevo. La seguridad me la tomé en serio desde el principio porque no quería que nadie piratease mi app y viera dónde estaban todos los buses. Puse JWT, cifrado en MQTT, validaciones de coordenadas raras.

Después hice pruebas reales. Conseguí a 5 amigos conductores, les instalé la app. Me fui a mi casa y veía cómo se movían los puntitos en el mapa en tiempo real. La primera vez que funcionó, literal salté. Dos semanas con ellos dándome feedback. Uno me dijo que le gustaba que no fuera complicada. Otro que cuidara la batería. Un pasajero amigo me dijo que la app le salvó porque finalmente supo que su bus iba retrasado y se fue caminando a otro lado.

Metí a 15 personas a probar. No todas las pruebas fueron perfectas. Una vez se crasheó por un bug con coordenadas inválidas. Otra vez una conductora dijo que le drenaba mucho la batería. Un admin quería dashboards más complejos. Eso me llevó a arreglar cosas, cambiar otras. No fue un proceso lineal.

La privacidad me importó desde el principio. No quería grabar dónde estaban los buses a cada segundo forever. Puse retención de datos, anonimización de historiales. Un profesor que me ayudó me dijo que tuviera cuidado con regulaciones de privacidad - todavía no sé si cumplí 100% pero lo intenté.

Tengo limitaciones claras. Trabajé solo. No hice testing con 1000 usuarios al mismo tiempo. No tengo una empresa atrás. No puedo hacer marketing. Pero precisamente eso me obligó a priorizar. Si algo no funcionaba, lo arreglaba. Si algo no era importante, lo dejaba para después. El resultado es algo que funciona, que es un poco desordenado en algunos lados, pero que resuelve el problema. Yo creo que eso es mejor que algo perfecto que nunca se termina.