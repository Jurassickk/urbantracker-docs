\section{Implementación}

La implementación de UrbanTracker se basa en una arquitectura distribuida que integra MQTT para la comunicación eficiente de datos GPS. La aplicación móvil del conductor actúa como publicador, capturando coordenadas de ubicación mediante el GPS del dispositivo. Estas coordenadas se envían al broker MQTT (Mosquitto) a través de un tópico específico, como "urbantracker/vehicle/{id}/location", donde {id} identifica al vehículo. El broker, alojado en la nube, recibe y gestiona estos mensajes de manera centralizada, distribuyéndolos a los suscriptores interesados, que incluyen aplicaciones web y móviles de usuarios finales y plataformas de administración.

El funcionamiento de MQTT en este contexto se basa en su modelo Pub/Sub: los publicadores envían mensajes sin conocer a los receptores, y los suscriptores se registran en tópicos relevantes. Por ejemplo, un usuario consultando una ruta específica se suscribe al tópico correspondiente, recibiendo actualizaciones en tiempo real sin polling constante. Los niveles de QoS (0: al menos una vez, 1: exactamente una vez, 2: asegurada) garantizan la entrega de mensajes incluso en redes inestables, almacenando temporalmente datos si la conexión se interrumpe. Esta implementación asegura una comunicación ágil y de baja latencia, esencial para la visualización continua de posiciones vehiculares.

\begin{figure}[h]
\centering
\includegraphics[width=0.5\textwidth]{./graphics/Imagen-10.jpg}
\caption{Imagen correspondiente al artículo [10]}
\end{figure}

\subsection{Componentes del Sistema}

El sistema UrbanTracker se estructura en tres componentes principales interconectados mediante MQTT, cada uno diseñado para optimizar la geolocalización y gestión de rutas en transporte público:

\begin{enumerate}
\item \textbf{Aplicación del Conductor (Publicador)}: Desarrollada para dispositivos móviles Android/iOS, esta aplicación accede al GPS nativo para capturar coordenadas de latitud, longitud y timestamp. Utiliza bibliotecas MQTT para publicar mensajes ligeros al broker, con intervalos configurables (ej. cada 10-30 segundos) para balancear precisión y consumo de batería. Incluye autenticación básica para asegurar que solo conductores autorizados publiquen datos.

\item \textbf{Broker MQTT (Mosquitto)}: Implementado con Mosquitto, un broker de código abierto, actúa como el núcleo de la comunicación. Gestiona conexiones persistentes, autentica publicadores y suscriptores, y distribuye mensajes a través de tópicos jerárquicos. Su arquitectura ligera permite manejar miles de conexiones simultáneas con bajo uso de recursos, soportando QoS para garantizar entrega en redes inestables. En la nube, asegura escalabilidad y disponibilidad alta.

\item \textbf{Plataforma Web/Móvil (Suscriptores)}: Incluye una interfaz web responsiva y aplicaciones móviles que permiten a usuarios suscribirse a tópicos específicos (ej. rutas de autobuses). Recibe actualizaciones en tiempo real para visualizar posiciones en mapas interactivos, estimar tiempos de llegada y gestionar rutas. Para administradores, ofrece dashboards para monitoreo de flotas, alertas de desviaciones y análisis de rendimiento, facilitando decisiones operativas.
\end{enumerate}

\subsection{Garantía del Tiempo Real y Eficiencia}

El uso de MQTT en la capa de transporte asegura que el requisito de visualización en tiempo real se cumpla con una performance de baja latencia. Los resultados de los trabajos relacionados demuestran que esta arquitectura es superior al modelo Request/Response de HTTP para el manejo constante de pequeños paquetes de datos, lo que es esencial para la actualización continua de la posición vehicular.

\subsection{Solución de Bajo Costo y Adaptabilidad}

Al validar que MQTT es la tecnología preferida para dispositivos con recursos limitados, se respalda la decisión de utilizar el dispositivo móvil del conductor como publicador de datos o como sensor de ubicación. Esto mantiene la solución de UrbanTracker en un bajo costo de implementación, cumpliendo el objetivo de ser accesible para cualquier entidad de transporte público o urbano, sin la necesidad de alguna costosa inversión en hardware especializado.

Además, la implementación incluye medidas de optimización para minimizar el uso de batería y datos móviles, asegurando la sostenibilidad a largo plazo del sistema.