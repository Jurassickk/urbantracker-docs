\section{Conclusiones}

La revisión exhaustiva de los 15 trabajos de investigación confirma de manera inequívoca que MQTT es el componente arquitectónico más idóneo para UrbanTracker en entornos de alta frecuencia y baja latencia. A diferencia de protocolos como HTTP, que requieren conexiones TCP completas para cada transmisión, MQTT mantiene conexiones persistentes, eliminando la sobrecarga de establecimiento y cierre, lo que reduce drásticamente la latencia en actualizaciones de posición.

El modelo Pub/Sub de MQTT, con su naturaleza ligera, asegura eficiencia energética y bajo consumo de datos, permitiendo que dispositivos móviles de conductores operen durante jornadas completas sin agotar baterías. Los resultados experimentales validan esta superioridad, mostrando latencias inferiores a 2 segundos y un consumo de batería del 5% por hora, superando alternativas tradicionales.

La validación de MQTT en proyectos de rastreo IoT, incluyendo geolocalización vehicular, refuerza su idoneidad para UrbanTracker. Su capacidad para manejar redes inestables mediante QoS garantiza que datos críticos de ubicación se entreguen incluso en zonas de baja cobertura, con Mosquitto asegurando recuperación automática de conexiones interrumpidas.

La integración de esquemas de seguridad, como TLS, es esencial para proteger la integridad de los datos GPS, fomentando la confianza en el sistema. En conjunto, estos atributos consolidan a MQTT como la base técnica de UrbanTracker, ofreciendo una solución robusta, eficiente y accesible para el transporte público urbano.
