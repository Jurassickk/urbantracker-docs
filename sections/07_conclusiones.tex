\section{Conclusiones}

La visualización de los sistemas investigados de los 15 trabajos de investigación confirma de manera asertiva que el protocolo MQTT es el componente arquitectónico más adecuado y esencial para la plataforma UrbanTracker en escenarios de alta frecuencia y baja latencia, ya que cada actualización de posición requeriría establecer, transmitir y cerrar una conexión, lo cual consume una latencia considerable.

El uso de MQTT elimina esta sobrecarga. Basándose en el modelo de Publicación/Suscripción en donde el publicador (el móvil) simplemente envía el mensaje ligero y ahí es donde actúa la naturaleza ligera de MQTT garantizan la eficiencia, el bajo consumo de recursos y la baja latencia, lo que permite a UrbanTracker cumplir su promesa de geolocalización en tiempo real.

La aprobación de múltiples proyectos investigados de rastreo e tracking IoT y la gran capacidad para operar en redes inestables o de baja latencia lo convierten en la opción ideal para la transmisión de datos de geolocalización desde el GPS de los dispositivos móviles de los conductores, consolidando a UrbanTracker como una solución accesible y escalable.

La capacidad de MQTT de mantener la conexión activa con un mínimo de disponibilidad, la posibilidad de que un vehículo pase por una zona de baja cobertura no podría significar la pérdida de datos, sino un retraso temporal, ya que el broker Mosquitto garantiza la entrega tan pronto como la red se vuelva a recuperar y esté estable.

Finalmente, la necesidad identificada en este proyecto y documento de reforzar la seguridad será la clave para la correcta implementación, para estar garantizando no solo la eficiencia sino también la confianza del usuario y la integridad de la información de datos.

En definitiva, la exhaustiva revisión de la extensa investigación se puede confirmar como el protocolo MQTT no es simplemente la mejor opción para UrbanTracker, sino que también es de las únicas tecnologías viables de alta demanda de rendimiento en tiempo real con las restricciones de recursos. La integración de MQTT con el debido refuerzo de seguridad es el factor determinante que transforma y valida la solución operativa y técnica que quiere llegar UrbanTracker a implementar.

\subsection{Trabajo Futuro}

Las áreas de desarrollo futuro incluyen:

\begin{itemize}
\item Implementación de esquemas de cifrado avanzados
\item Desarrollo de algoritmos de optimización de rutas
\item Integración con sistemas de pago móvil
\item Expansión a otros modos de transporte público
\end{itemize}