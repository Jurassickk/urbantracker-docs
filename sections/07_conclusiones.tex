\section{Conclusiones}
UrbanTracker cumple de manera efectiva los objetivos planteados al inicio del proyecto, demostrando que es posible mejorar tanto la experiencia del usuario como la operación del servicio público de transporte mediante el uso de tecnologías de geolocalización en tiempo real. La plataforma logró integrar de forma estable distintos componentes —aplicación móvil, interfaces web, backend modular y mensajería MQTT— para ofrecer un sistema capaz de reflejar con precisión la ubicación de los vehículos y proporcionar herramientas útiles para la gestión de la flota. Esta integración permitió validar que el enfoque técnico seleccionado es adecuado para entornos urbanos que requieren soluciones ágiles, de bajo costo y con capacidad de actualización constante.

Los resultados obtenidos en las pruebas confirman la viabilidad de la arquitectura híbrida, combinando monolito modular con protocolos IoT, lo que facilita la escalabilidad futura. La experiencia de desarrollo resalta la importancia de una metodología iterativa para identificar y resolver problemas tempranos, asegurando un producto robusto desde las primeras fases.

\subsection{Trabajo futuro}
A pesar de los buenos resultados, se reconoce que el desarrollo del sistema aún puede fortalecerse mediante la consolidación de pruebas unitarias e interfaces más completas. Un mayor énfasis en la automatización de pruebas ayudaría a mejorar la estabilidad general del software y a identificar posibles fallos antes de escalar a entornos con mayor número de usuarios o vehículos. Asimismo, la incorporación de modelos de análisis predictivo —como estimaciones de tiempo de llegada o detección temprana de desvíos de ruta— representa una línea de trabajo con alto potencial para incrementar el valor del sistema, especialmente desde la perspectiva del usuario final.

Otro aspecto importante a considerar en futuras iteraciones es la privacidad. Aunque la seguridad básica mediante autenticación JWT ha sido implementada, la gestión del historial de ubicaciones, el cifrado de datos en tránsito y las políticas de retención deben estudiarse con mayor detalle para garantizar que la información sensible sea tratada adecuadamente. Estas mejoras no solo aumentarían la confianza de los usuarios, sino que también facilitarían la adopción del sistema en entornos donde las regulaciones en protección de datos son más estrictas.

Además, se plantea integrar inteligencia artificial para optimizar rutas dinámicamente, considerando factores como tráfico en tiempo real y condiciones climáticas. La expansión a otras modalidades de transporte, como bicicletas compartidas o scooters eléctricos, podría ampliar el alcance del sistema. Finalmente, colaboraciones con entidades gubernamentales para estandarizar APIs de transporte público facilitaría la interoperabilidad entre ciudades.

Finalmente, la experiencia adquirida durante el desarrollo de UrbanTracker constituye un punto de partida valioso para futuras implementaciones en otros contextos urbanos o de transporte. La arquitectura modular, la flexibilidad del diseño y las tecnologías seleccionadas permiten adaptar el sistema a diferentes tipos de flotas o necesidades institucionales. En conjunto, el proyecto sienta bases sólidas para continuar evolucionando hacia soluciones más completas e inteligentes de movilidad urbana.
Además, la experiencia adquirida durante el desarrollo de UrbanTracker constituye un punto de partida valioso para futuras implementaciones en otros contextos urbanos o de transporte. La arquitectura modular, la flexibilidad del diseño y las tecnologías seleccionadas permiten adaptar el sistema a diferentes tipos de flotas o necesidades institucionales, como sistemas de transporte escolar, logística urbana o monitoreo de vehículos de emergencia.

En resumen, UrbanTracker no solo resuelve un problema técnico, sino que contribuye a una transformación más amplia en la movilidad urbana, promoviendo sistemas de transporte más eficientes, sostenibles y centrados en el usuario. Su implementación exitosa valida el potencial de las tecnologías modernas para mejorar la calidad de vida en entornos urbanos complejos.