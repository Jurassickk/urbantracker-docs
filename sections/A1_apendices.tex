\appendix
\section{Checklist de reproducibilidad (plantilla)}
\begin{itemize}[leftmargin=*]
  \item \textbf{Repositorio y commits}: documentar las ramas de backend (Spring Boot) y frontend (React Native / Next.js), junto con el hash utilizado para cada experimento.
  \item \textbf{Entorno Docker}: registrar versiones de Docker y Docker Compose, además de las variables de entorno empleadas para PostgreSQL y Mosquitto.
  \item \textbf{Configuración del backend}: detallar perfiles de Spring Boot, credenciales JWT temporales y scripts de creación de esquemas por módulo.
  \item \textbf{Aplicación móvil}: especificar versiones de React Native, dependencias instaladas y pasos para habilitar permisos de localización en los dispositivos de prueba.
  \item \textbf{Escenarios de prueba}: enumerar los recorridos simulados, la frecuencia de publicación y los criterios de aceptación (latencia máxima, porcentaje de entregas exitosas).
  \item \textbf{Datos generados}: conservar los historiales de posiciones exportados, capturas del panel web y cualquier métrica adicional utilizada para evaluar el desempeño.
\end{itemize}

\section{Agradecimientos}
Los autores agradecen al equipo de desarrollo de UrbanTracker por su dedicación en las distintas fases del proyecto, en especial a Diego F. Cuellar por sus valiosa contribuciones de programación y pruebas. Asimismo, se extiende nuestro agradecimiento al Servicio Nacional de Aprendizaje (SENA) y sus instructores por el apoyo institucional y técnico brindado durante la realización de este trabajo. Su orientación y recursos fueron fundamentales para alcanzar los objetivos propuestos.

\section{Contribuciones de autor}
\textbf{Brayan Estiven Carvajal Padilla:} Conceptualización del proyecto; análisis de requisitos; diseño de interfaces; desarrollo del frontend; realización de pruebas; redacción inicial del manuscrito.
\textbf{Jesús Ariel González Bonilla:} Supervisión general del proyecto; mentoría metodológica durante la investigación.