\appendix
\section{Checklist de reproducibilidad (plantilla)}
\begin{itemize}[leftmargin=*]
  \item \textbf{Datos}: fuente, versión, licencias, anonimización.
  \item \textbf{Código}: repositorio, commit hash, instrucciones de ejecución.
  \item \textbf{Entorno}: SO, versión de compiladores, dependencias, semillas.
  \item \textbf{Procedimiento}: pasos exactos para replicar resultados.
  \item \textbf{Resultados}: tablas/figuras generadas automáticamente en \texttt{build/}.
\end{itemize}
\section{Referencias}

DESARROLLO DE UN SISTEMA DE LOCALIZACIÓN BASADO EN GPS E IOT: UN ESTUDIO DE CASO EN QUITO (2025). Google Sschoolar. Disponible en: https://www.investigacionistct.ec/ojs/index.php/investigacion_tecnologica/article/view/165

SISTEMA DE GEOLOCALIZACION DE VEHICULOS RECOLECTORES DE BASURA APLICANDO INTERNET DE LAS COSAS (2025). Google Sschoolar. Disponible en: https://repositorio.upea.bo/jspui/handle/123456789/89

Aplicación web para el control de desviaciones de rutas en el transporte publico mediante IOT (2025). Google Sschoolar. Disponible en: https://www.dspace.espol.edu.ec/handle/123456789/65811

Aplicación del protocolo MQTT y recolección de datos para aplicaciones IoT (2025). Google Sschoolar. Disponible en: https://repositorio.unitec.edu/items/2b823853-4bb8-4fd7-887a-d93f122933c5

Arquitectura orientada a eventos sobre protocolo MQTT (2025). Google Sschoolar. Disponible en: https://sedici.unlp.edu.ar/handle/10915/130301

Especificación de Requisitos de Software (SRS) – UrbanTracker, versión 1.0 (2025). 

Aplicación del protocolo mqtt y recolección de datos para aplicaciones iot. (2025). [Artículo]. Repositorio UNITEC. https://repositorio.unitec.edu/items/2b823853-4bb8-4fd7-887a-d93f122933c5

Aplicación web para el control de desviaciones de rutas en el transporte publico mediante iot. (2025). [Tesis]. Repositorio Digital ESPOL. https://www.dspace.espol.edu.ec/handle/123456789/65811

Arquitectura orientada a eventos sobre protocolo mqtt. (2025). [Tesis]. SEDICI, Universidad Nacional de La Plata. https://sedici.unlp.edu.ar/handle/10915/130301

Desarrollo de un sistema de localización basado en gps e iot: un estudio de caso en quito. (2025). [Artículo]. Investigación Tecnológica (ISTCT). https://www.investigacionistct.ec/ojs/index.php/investigacion_tecnologica/article/view/165

Sistema de geolocalización de vehículos recolectores de basura aplicando internet de las cosas. (2025). [Artículo]. Repositorio UPEA. https://repositorio.upea.bo/jspui/handle/123456789/89