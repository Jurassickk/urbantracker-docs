\section{Limitaciones y desafíos futuros}

Ahora, siendo completamente honesto, UrbanTracker no es perfecto. Hay varias limitaciones que debo admitir y que quiero mejorar en futuras versiones.

La más obvia es la dependencia del GPS y la señal móvil. En ciertas zonas de Bogotá - especialmente donde hay edificios súper altos o túneles - la precisión del GPS puede ser problemática. He visto casos donde la app muestra un bus en el piso 15 de un edificio porque el GPS se confundió. Y obviamente, si no hay señal móvil, no hay actualizaciones en tiempo real.

\begin{figure}[h]
\centering
\includegraphics[width=0.8\textwidth]{graphics/8-img.png}
\caption{Limitaciones técnicas de UrbanTracker: zonas sin cobertura GPS y problemas de precisión en áreas urbanas densas}
\label{fig:limitaciones-tecnicas}
\end{figure}

Otra limitación práctica es la batería. Aunque intenté optimizar el consumo, la app sigue consumiendo batería. Algunos conductores se han quejado de que al final del día el celular les queda con poca batería. Es un problema real que necesito resolver.

También tengo que admitir que las pruebas de escalabilidad fueron limitadas. Aunque simulé hasta 500 vehículos, la verdad es que no he probado el sistema con flotas reales de ese tamaño. En mis pruebas caseras con 10 teléfonos, todo funcionaba perfecto. Pero ¿qué pasa cuando tienes 200 conductores usando la app al mismo tiempo? ¿El broker MQTT se va a soportar? Estas son preguntas que realmente necesito responder con pruebas en el mundo real.

Además, la base de datos PostgreSQL podría necesitar optimizaciones si el número de ubicaciones crece exponencialmente. Actualmente guardamos cada ubicación que envía un conductor. En una flota grande, eso podría significar millones de registros por día. No he explorado aún estrategias como bases de datos distribuidas o sistemas de streaming para manejar esos volúmenes.

Desde seguridad y privacidad, aunque implementamos JWT y cifrado básico, faltan medidas avanzadas. Historial podría ser vulnerable a inferencia sin anonimización, rastreando movimientos individuales. En regulatorios estrictos como RGPD, necesitamos consentimiento explícito y retención rigurosa. Autenticación no contempla robo de credenciales o ataques intermediarios.

En usabilidad, interfaces optimizadas para Android, no probadas en otros OS o tamaños. Usuarios con discapacidades enfrentarían dificultades sin accesibilidad como lectores de pantalla. Multilingüe limitado a español, restringiendo adopción multicultural.

Desafíos futuros incluyen integración con ciudades inteligentes, como semáforos o movilidad compartida. Requiere estándares abiertos y APIs. IA para predicciones mejoraría precisión 20-30\%, pero introduce complejidades en entrenamiento y sesgos.

Finalmente, sostenibilidad económica es clave. Reduce costos operativos optimizando rutas, pero implementación inicial requiere inversión en hardware y capacitación. En municipios pobres, adopción dependería de subsidios. Evolución a código abierto fomentaría colaboraciones, pero necesita estrategia para contribuciones.

En resumen, UrbanTracker establece base sólida, pero éxito futuro depende de abordar limitaciones con iteraciones, colaboraciones y evaluación de impactos. Fortalecerá robustez y expandirá aplicabilidad a urbanos diversos.
Además de la dependencia GPS, enfrentamos desafíos con la precisión en zonas densas. En áreas como el centro histórico de Bogotá, donde edificios altos interfieren con señales satelitales, la ubicación puede desviarse hasta 50 metros. Esto afecta la confiabilidad para usuarios que necesitan precisión exacta, como personas con movilidad reducida que dependen de paradas específicas.

Otro aspecto crítico es la gestión de batería en dispositivos antiguos. Aunque optimizamos el consumo, conductores con teléfonos de gama baja reportan agotamiento después de 6-7 horas de uso continuo. Esto no solo afecta la operatividad, sino que también genera resistencia entre operadores que temen costos adicionales de reemplazo de baterías.

Desde lo técnico, la escalabilidad de PostgreSQL requiere atención. Con flotas grandes, consultas de ubicación histórica pueden volverse lentas sin índices adecuados. No implementamos particionamiento de tablas, lo que podría causar problemas en entornos con millones de registros diarios.

La seguridad también presenta brechas. Aunque usamos JWT, no incluimos rotación automática de claves ni detección de anomalías en accesos. En escenarios de ciberataques, esto podría comprometer datos sensibles, especialmente considerando que ubicaciones pueden inferir patrones de movimiento personales.

Usabilidad limita adopción multicultural. La interfaz en español asume familiaridad con términos técnicos; usuarios de otras culturas o con bajo alfabetismo digital enfrentan barreras. Pruebas con grupos diversos revelaron confusiones en iconos y navegación, sugiriendo necesidad de diseños más universales.

Económicamente, el costo inicial de capacitación es subestimado. Administradores requieren entrenamiento no solo en uso del sistema, sino en interpretación de datos. Sin soporte continuo, el valor del sistema se reduce, especialmente en municipios con recursos limitados.

Finalmente, la dependencia de conectividad móvil excluye áreas rurales o con cobertura deficiente. En rutas suburbanas de Bogotá, donde señal 4G es intermitente, el sistema falla, dejando usuarios sin información alternativa. Esto resalta la necesidad de modos offline o integraciones con redes satelitales para cobertura universal.