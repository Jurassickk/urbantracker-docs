\section{Limitaciones y desafíos futuros}
Aunque UrbanTracker ha demostrado ser una solución efectiva para el seguimiento en tiempo real de vehículos de transporte público, existen varias limitaciones que deben abordarse en futuras iteraciones para maximizar su potencial y adaptabilidad. Una de las principales limitaciones técnicas radica en la dependencia de la conectividad GPS y de red móvil. En áreas urbanas densas o con infraestructura deficiente, como túneles, zonas de sombra de señal o edificios altos, la precisión de las coordenadas puede verse comprometida, lo que resulta en actualizaciones inexactas o retrasos en la transmisión de datos. Este problema se agrava en dispositivos móviles con baterías limitadas, donde el uso continuo del GPS puede agotar rápidamente la energía, afectando la viabilidad operativa para conductores que dependen de sus teléfonos personales.

Otra limitación significativa es la escalabilidad del sistema. Aunque el diseño modular permite una transición hacia microservicios, las pruebas actuales se han centrado en escenarios de pequeña escala con un número reducido de vehículos. En entornos con flotas más grandes —como ciudades con cientos o miles de autobuses— el broker MQTT podría enfrentar cuellos de botella si no se implementan estrategias de particionamiento de tópicos o balanceo de carga. Además, el procesamiento en tiempo real de grandes volúmenes de datos requiere optimizaciones adicionales, como el uso de bases de datos distribuidas o motores de análisis en streaming, que no han sido explorados en esta fase inicial.

Desde una perspectiva de seguridad y privacidad, aunque se han implementado mecanismos básicos como JWT y cifrado en tránsito, el sistema aún carece de medidas avanzadas para proteger datos sensibles. Por ejemplo, el historial de ubicaciones podría ser vulnerable a ataques de inferencia si no se anonimizan adecuadamente, permitiendo rastrear patrones de movimiento de usuarios individuales. En contextos regulatorios estrictos, como el RGPD en Europa o leyes similares en otros países, sería necesario integrar controles de consentimiento explícito y políticas de retención de datos más rigurosas. Además, la autenticación actual no contempla escenarios de robo de credenciales o ataques de intermediario en redes públicas, lo que podría comprometer la integridad del sistema.

En términos de usabilidad, las interfaces actuales están optimizadas para dispositivos Android, pero no han sido probadas exhaustivamente en otros sistemas operativos o tamaños de pantalla. Los usuarios con discapacidades visuales o motoras podrían enfrentar dificultades con la navegación, ya que no se han incorporado características de accesibilidad como lectores de pantalla o controles por voz. Asimismo, la experiencia multilingüe se limita al español, lo que restringe su adopción en contextos multiculturales.

Los desafíos futuros incluyen la integración con plataformas de ciudades inteligentes, como sistemas de semáforos adaptativos o aplicaciones de movilidad compartida. Esta interoperabilidad requeriría estándares abiertos y APIs estandarizadas, lo que plantea desafíos técnicos y regulatorios. Otro aspecto prometedor es la incorporación de inteligencia artificial para predicciones predictivas, como estimaciones de llegada basadas en aprendizaje automático, que podrían mejorar la precisión en un 20-30\% según estudios preliminares. Sin embargo, esto introduce complejidades en el entrenamiento de modelos con datos históricos y la gestión de sesgos en los algoritmos.

Finalmente, la sostenibilidad económica del sistema es un desafío clave. Aunque UrbanTracker reduce costos operativos al optimizar rutas, la implementación inicial requiere inversiones en hardware, capacitación y mantenimiento. En municipios con presupuestos limitados, la adopción podría depender de subsidios o modelos de financiamiento público-privado. A largo plazo, la evolución hacia un sistema de código abierto podría fomentar colaboraciones globales y acelerar innovaciones, pero requiere una estrategia clara para gestionar contribuciones comunitarias y asegurar la calidad del software.

En resumen, mientras UrbanTracker establece una base sólida para el seguimiento vehicular en tiempo real, su éxito futuro dependerá de abordar estas limitaciones mediante iteraciones continuas, colaboraciones interdisciplinarias y una evaluación rigurosa de impactos sociales, técnicos y económicos. Estas mejoras no solo fortalecerán la robustez del sistema, sino que también expandirán su aplicabilidad a contextos urbanos diversos y exigentes.