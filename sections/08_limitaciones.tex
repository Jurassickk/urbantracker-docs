\section{Limitaciones y desafíos futuros}

Oof, acá viene lo que no quería admitir pero tengo que. UrbanTracker funciona - sí, funciona - pero hay un montón de cosas que no salieron como pensé.

El GPS es lo más obvio. Y lo más frustrante. En teoría es simple: preguntás al satélite, te da coordenadas, listo. La realidad es que el GPS es impredecible. En Neiva centro, cerca del edificio Telecom, la torre de servicios, la zona bancaria, se vuelve completamente loco. A veces desviaciones de 20 metros. Otro día 150. No sé por qué. Tal vez los edificios, la humedad, quién sabe. Una vez mostró un bus adentro del edificio. Literalmente flotando. Mi hermana estaba esperando y la app le decía "2 minutos" y nunca llegó. Espero 20, 25. Después me llamó tipo "¿qué estás haciendo?". Se fue caminando. El bus pasó 15 minutos después pero ya no le importaba. Eso pasó varias veces. Sin señal móvil la cosa es aún peor - los buses desaparecen del mapa completamente. Parece que no existieran. Y en el campo, zonas rurales, es directamente imposible. No hay cobertura, el GPS es inútil. No tengo solución para eso. Es un problema que sé que existe pero no puedo arreglarlo ahora.

La batería. Dios. Eso me quita el sueño. Implementé tres - no, cuatro - cosas diferentes. Reducir frecuencia del GPS, idle cuando está estacionado, modo de ahorro manual. Nada. Los conductores siguen con el teléfono muerto a mitad del turno. Mi primo que maneja la ruta Chipre me dejó de hablar por casi dos semanas después de que se le apagó el teléfono. Tenía dos horas más de turno. Se quedó tirado sin poder llamar a su casa. Eso fue completamente mi culpa. No hay excusa. Y no es que los conductores usen iPhones nuevos - son Redmi viejos, Samsung J2, aparatos que duran máximo 5-6 horas si tenés suerte. Un conductor necesita que dure 10, 12 horas. A veces más. No lo logré. Cada vez que alguien menciona que se le murió la batería me siento mal.

Escalabilidad. Bueno, acá el problema es que nunca la probé de verdad. Simulé 500 vehículos en mi laptop, funcionó perfecto. Pero simulación no es realidad. Con teléfonos reales, probé con 5. Después con 10. Mi tío y sus amigos se reían viéndome en la calle con como 15 celulares viejos en una mochila. Todos enviando GPS. Funcionó. Pero eso no me dice qué pasa con 200 conductores, 500, 1000. MQTT es robusto pero hay un límite. No sé dónde está ese límite. Podría ser 300. Podría ser 5000. No lo sé. La base de datos seguro empieza a ralentizarse en algún momento. El servidor podría no dar abasto. Para saber necesitaría una prueba real con una flota grande, pero eso cuesta dinero que no tengo.

PostgreSQL. Guardamos cada punto GPS cada 10-15 segundos. Eso son 240-360 puntos por hora por conductor. Con 200 conductores en un mes... millones de registros. Muchísimos. Con buenos índices anda. Pero cuando hago queries complejas se ralentiza. Nunca me metí en sharding, distributed databases, eso es otro nivel. Está ahí como deuda técnica, esperando. Talvez cuando crezca lo arregla alguien más.

Seguridad. Acá duele. JWT implementé pero es básico. No rotan los tokens. Si alguien en la ruta roba el teléfono de un conductor, accede a todo el historial de ubicaciones. Eso es grave. Dónde vivía el tipo, dónde iba después del turno, todo. MQTT tiene cifrado pero no end-to-end. Un administrador de la base podría verlo si quisiera. Las contraseñas las hasheé con bcrypt pero nada del otro mundo. Un hacker serio que se lo proponga podría entrar. GDPR no estaría contento. Honestamente? No puedo decir que la privacidad esté garantizada. No al 100\%.

Solo Android. Fin de la historia. iOS no existe en mi mundo. Web tampoco. Un conductor con iPhone está afuera. Alguien con discapacidad visual? No pensé en eso. No hay screen reader, no hay nada accesible. Es falta mía. Y sordo? Necesitaría alertas de audio que no tengo. Idioma: español. Punto. Un conductor de otra provincia que hable otro idioma está perdido.

Zonas sin cobertura no funcionan. Si trabajás una ruta que va a pueblos donde no hay 4G, la app es inútil. Modo offline sería la solución - guardar GPS localmente, sincronizar después cuando hay conexión. Nunca lo hice. Era trabajo extra que no prioricé.

Economía. Sí, es más barato que sistemas tradicionales. Para empresas grandes. Pero implementarlo cuesta. Equipos nuevos, entrenamiento de gente, infraestructura. Un municipio pobre no tiene plata. Habría que subsidiar. ¿Quién paga eso? Yo no puedo solo. Poner el código en GitHub es una opción pero depende de que otros lo quieran mejorar. Y no todos los municipios pueden trabajar con código abierto.

El futuro me atrae. Conectar con sistemas de ciudades inteligentes. APIs con semáforos inteligentes, datos de tráfico. Machine learning para predecir embotellamientos y sugerir rutas alternativas. Mejoraría precisión. Quizás 30-35\%, talvez menos, no sé. Es lindo en teoría. En práctica necesita muchísimos datos, computación pesada, recursos que no tengo.

En conclusión: construí algo que anda para el caso que probé. 15 conductores, Neiva, celulares medios. Afuera la realidad es más complicada. Flotas reales son más grandes. Los dispositivos son viejos. La conectividad es de mierda. Los presupuestos son más apretados. UrbanTracker v1 es un proof of concept. Funciona. V2 tiene que arreglar los problemas que vi acá. Va a ser trabajo. Pero sé qué necesito hacer.