\section{Casos de estudio y experimentos}

Para saber si UrbanTracker realmente funcionaba o si era solo algo que andaba en mi compu, necesitaba probarlo en la calle. Con gente real. Buses reales. No en simulaciones, eso es fácil. Cualquiera puede hacer funcionar algo en simulaciones.

El primer experimento fue simple. Cogí la ruta 7 de Neiva - la que va de Éxito hasta Tierra Santa - 15 paradas más o menos, no conté bien. Puse a mi primo a conducir con mi app abierta, actualizando GPS cada 30 segundos. Yo iba siguiendo en otra compu desde mi casa. El miedo que tenía era que se cayera todo después de 5 minutos. No pasó. La precisión fue de 95\% incluso cuando pasaba bajo puentes o en zonas donde la señal se iba. MQTT se reconectaba automático. Eso me sorprendió. En serio. Pensé "ok, talvez esto funciona" pero no estaba seguro si era suerte o si realmente andaba bien.

Después quise ver qué pasaba si metía muchos buses al mismo tiempo. Simulé 50 vehículos en mi servidor. Esperaba que se rompiera. Spring Boot aguantó. Las respuestas estaban rápido - menos de 500 ms, creo. No medí exacto pero se sentía rápido. La memoria no explotó. Eso fue un alivio porque si no funcionaba con 50 nunca iba a funcionar con 200. O talvez sí, no sé. Las simulaciones no te dicen todo.

Pero simulaciones son una cosa. Gente de verdad usando la app es otra completamente diferente. Metí 20 usuarios - amigos, gente de la familia, conductores que conocía, gente random que me pidió que la probara. Les dije "usen esto, díganme qué piensan, sean honestos". El 85\% dijo que era fácil de usar. Que se entendía rápido. Algunos dijeron que la app se cargaba lenta en teléfonos viejos - y tenían razón, cargaba lenta. Un tipo ciego me pidió que pusiera audio para que él pudiera saber dónde estaba el bus. No lo hice. Eso fue un fallo mío. Debería haberlo hecho pero no me dio tiempo. O bueno, sí me dio tiempo pero no lo prioricé. No es excusa.

Lo que pasó después fue que tuve la oportunidad de trabajar con una empresa real. La empresa de buses Chipre - la misma donde trabaja mi tío. El gerente me conocía, no sé bien por qué. Me dijo "ok, ponelo en 5 rutas durante dos semanas, a ver qué pasa". Eso fue septiembre. Probablemente. Fue hace un año, no recuerdo bien el mes. Literalmente me temblaban las manos. Si se rompía, era un fracaso público. Mi tío estaría ahí viéndolo fallar. Eso me angustiaba.

Las dos primeras semanas fueron caóticas. Muy caóticas. Un conductor no sabía cómo prender el teléfono. Pasé dos horas explicándole. Otro se le cayó y se le rompió la pantalla - eso fue mi culpa por haber elegido teléfonos viejos. Un tercero dijo que le drenaba la batería - probablemente tenía razón, la batería es un problema. Pero la app funcionó. Todos los días. Sin crashes grandes. Hubo un crash una vez pero fue porque el usuario instaló otra app y eso rompió algo. Bueno, no sé si fue eso pero probablemente. Los conductores empezaron a usarla. Los usuarios de los buses también descargaban mi app y checaban cuándo llegaba su bus. Eso fue cool.

A la mitad de la prueba, algo cambió que no esperé. Los conductores se dieron cuenta de que con la app podían hacer rutas más eficientes. Un tipo me dijo "che, si voy por acá en lugar de por acá llego 3 minutos antes". Eso no estaba en mi código - él solo descubrió una ruta mejor viendo el mapa. Mi app simplemente le mostró el camino. Eso fue genial porque significa que la gente usaba el sistema de formas que no había anticipado. Positivamente, supongo.

Las quejas de retrasos bajaron 25\%. Eso es un número que puedo decir con seguridad. Antes de la app había 20-30 quejas por semana en esas 5 rutas. Después bajó a 15. Podría ser coincidencia, podría ser que la gente simplemente se acostumbró a esperar. Pero yo creo que fue porque la gente podía rastrear el bus y sabía que estaba yendo hacia ellos. O talvez simplemente porque sabían que el admin estaba viendo los datos en tiempo real y los conductores se comportaban mejor. No lo sé con certeza.

Un usuario me escribió a Facebook - sí, me encontró en Facebook, fue raro - y me dijo que desde que usaba mi app había reducido su tiempo de traslado en 15 minutos. Que antes llegaba 30 minutos antes a la parada "por si acaso". Ahora llegaba justo a tiempo. Eso es pequeño pero es real. Es una persona. Una vida un poco diferente.

Hice otro experimento donde metí 100 dispositivos simulados al mismo tiempo. Quería ver dónde se rompía. El sistema aguantó. Latencia bajo 300 ms. Pero vi que las queries a la base de datos se ponían lentas. Queries complejas tardaban más. Si tenías 500 dispositivos probablemente explotaba todo. Eso lo dejé como "lo arreglo después". Nunca lo arreglé.

También traté de integrar datos externos. Tenía una API de tráfico que me daba información de congestión en la ciudad. Intenté usarla para predecir mejor cuándo llegaban los buses. Mejoré las predicciones como 15\% durante horas pico. O talvez fue 10\%, no mido exacto. Pero también agregó complejidad. Más datos, más cálculos, más cosas que podían salir mal. Una vez se rompió la conexión con la API y la app quedó sin esos datos. Eso fue confuso.

Lo que más me sorprendió fue lo simple que los conductores encontraban la app. Un tipo de 55 años, educación básica, aprendió a usarla en 5 minutos. Presionaba el botón de "iniciar turno", el app rastreaba su ubicación, listo. Fin. No necesitaba saber cómo funcionaba GPS o MQTT. Solo necesitaba que funcionara. Y funcionó para él. Eso fue un alivio.

Pero también vi problemas constantemente. Un conductor se perdió porque la app le mostró una ruta que en la práctica no existía - estaba bloqueada por construcción. Otra vez un usuario esperó en la parada equivocada porque la app le mostró el bus llegando a una parada que estaba 200 metros al norte de donde realmente estaba. El usuario se molestó. Yo me molestí. Cosas pequeñas que son grandes cuando te afectan.

Un momento que me quedó grabado fue cuando mi abuela me llamó. Estaba usando la app en la parada de Éxito esperando el bus. Me dijo "el punto azul dice que faltan 4 minutos". Cuatro minutos después pasó el bus. Ella casi llora. O talvez sí lloró, no estoy seguro. "Funcionó", me dijo. "Tu cosa funcionó". Eso fue... bueno, eso fue mucho para mí. Honestamente.

Los datos que junté en esas pruebas mostraban que la cosa andaba. Pero también mostraban que no era perfecta. Había edge cases constantemente. Había gente para la cual no funcionaba bien. Había limitaciones técnicas que no podía resolver. Algunas porque era difícil. Otras porque no tenía tiempo. Otras porque simplemente no las vi venir.

Lo que aprendí fue que entre simulaciones en mi compu y gente real usando la app hay un abismo. Un abismo gigante. Las simulaciones te dicen si la arquitectura aguanta presión. Los usuarios reales te dicen si es útil. O si no te odia. Yo creía que mi app era buena hasta que un usuario sordo me pidió audio. Eso me hizo darme cuenta de que hay dimensiones del problema que no había pensado. Que no había considerado. Que simplemente no sabía que existían.

También aprendí que los números pueden mentir. 25\% menos quejas podría ser porque la app funciona o podría ser porque la gente simplemente dejó de quejarse porque sabía que era monitoreado. Los números son útiles pero no cuentan toda la historia.

\begin{figure}[h!]
    \centering
    \includegraphics[width=0.90\textwidth]{graphics/10-img.png}
    \caption{Evidencia de campo: capturas de pantalla de UrbanTracker en pruebas reales de funcionamiento mostrando interfaz web, aplicación móvil Android, panel administrativo con estado de flota en tiempo real, y detalle de visualización de ruta en mapa.}
    \label{fig:field-evidence}
\end{figure}

En conclusión, los experimentos confirmaron que UrbanTracker funciona. O funciona para algunas cosas. Para la gente que lo necesita en Neiva, funciona la mayoría del tiempo. No es perfecto. Tiene problemas. Tendrá más problemas que no veo. Pero funciona. Eso es suficiente para una v1. Talvez.