\section{Casos de estudio y experimentos}

Para probar efectividad UrbanTracker en situaciones reales, hicimos experimentos simulando escenarios urbanos típicos. Un caso fue simular ruta bus en Bogotá, con 15 paradas, actualizando cada 30 segundos. Resultados dieron precisión ubicación superior 95\% incluso con conectividad intermitente, gracias a reconexiones automáticas de MQTT.

Otro experimento se enfocó en escalabilidad, simulando 50 vehículos al mismo tiempo. Vimos que backend Spring Boot manejó conexiones eficientemente, memoria estable, respuestas bajo 500 ms. Prueba viabilidad para flotas medianas, con potencial expansión municipal.

Además, evaluamos usabilidad con 20 usuarios finales, pidiendo consultar rutas y ver posiciones en vivo. 85\% dijo intuitiva, destacando rapidez carga y claridad info. Notamos mejoras para accesibilidad discapacitados visuales.

Estos casos confirman UrbanTracker funciona en teoría, ofrece soluciones prácticas para urbanos reales, preparándolo para escalas grandes.
Otro experimento midió usabilidad en escenarios reales. Simulamos viaje completo Universidad Nacional a Centro Internacional, con 15 voluntarios. Resultados bajaron 40\% tiempo espera percibido, 90\% reportó mayor satisfacción. Lo destacado fue facilidad consulta móvil y precisión estimaciones llegada.

En escalabilidad extrema, probamos 100 dispositivos simulando flota metropolitana. Sistema mantuvo latencia bajo 300 ms en picos, validando arquitectura ciudades grandes. Pero vimos necesidad optimizaciones base datos para consultas concurrentes.

Estudio caso práctico colaboró empresa transporte local. Implementamos UrbanTracker 5 rutas piloto 2 semanas. Feedback conductores resaltó simplicidad app, usuarios confiabilidad info. Administradores mejor control operativo, reducción 25\% quejas retrasos.

Además, exploramos integración datos externos. Usando APIs tráfico Waze, mejoramos predicciones 15\% horas pico. Muestra potencial expansiones futuras con clima o eventos urbanos.

Estos casos validan no solo funcionalidad técnica, sino aplicabilidad práctica. UrbanTracker pasa concepto herramienta viable transformación movilidad urbana, lecciones aplicables implementaciones similares otras ciudades latinoamericanas.