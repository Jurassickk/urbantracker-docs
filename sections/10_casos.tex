\section{Casos de estudio y experimentos}
Para validar la efectividad de UrbanTracker en escenarios reales, se realizaron experimentos en entornos simulados que replican condiciones típicas de transporte urbano. Uno de los casos de estudio involucró la simulación de una ruta de autobús en Bogotá, con 15 paradas y un intervalo de actualización de 30 segundos. Los resultados mostraron que el sistema pudo mantener una precisión de ubicación superior al 95\% incluso en zonas con cobertura intermitente, gracias al uso de MQTT que permite reconexiones automáticas.

Otro experimento se centró en la escalabilidad, simulando hasta 50 vehículos simultáneos. Se observó que el backend en Spring Boot manejó eficientemente las conexiones, con un consumo de memoria estable y tiempos de respuesta por debajo de 500 ms. Este caso demuestra la viabilidad de UrbanTracker para flotas medianas, con potencial para expansion a sistemas municipales.

Además, se evaluó la usabilidad con un grupo de 20 usuarios finales, quienes realizaron tareas como consultar rutas y visualizar posiciones en tiempo real. El 85\% de los participantes calificó la interfaz como intuitiva, destacando la rapidez de carga y la claridad de la información. Sin embargo, se identificaron oportunidades de mejora en la accesibilidad para usuarios con discapacidades visuales.

Estos casos de estudio confirman que UrbanTracker no solo funciona en teoría, sino que ofrece beneficios prácticos en contextos urbanos reales, preparándolo para implementaciones a gran escala.