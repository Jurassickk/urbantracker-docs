El sistema UrbanTracker surge como una solución para mitigar la falta de información sobre las ubicaciones y rutas de los vehículos del sistema público urbano. El objetivo principal es desarrollar una plataforma de geolocalización y gestión para la visualización de los vehículos y la administración efectiva de rutas. Las dificultades principales incluyen garantizar la transmisión de datos de ubicación desde los dispositivos móviles de los conductores hacia el servidor con una baja latencia y finalmente a los usuarios de una manera ágil y eficiente. En este artículo se presenta una justificación arquitectónica basada en una revisión de quince estudios de casos que respalda el uso del protocolo MQTT (protocolo de envío ligero y eficiente) como la principal herramienta de comunicación en tiempo real. Los artículos relacionados demuestran un excelente seguimiento sobre que MQTT es un componente esencial para alcanzar el desarrollo de un sistema de seguimiento confiable, escalable y adaptado a entornos de recursos escasos.

Palabras clave: Transporte público; Geolocalización; MQTT; IoT; Comunicación en tiempo real; Rutas.