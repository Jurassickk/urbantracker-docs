La idea de UrbanTracker nació de una incomodidad que seguramente todos hemos sentido en algún momento. Durante años, llegué tarde a varios compromisos simplemente porque perdí el bus, cuando me atreví a preguntarles a otras personas en la parada, todos quedaron en la mata de “por ahí viene”. La incertidumbre rondaba durante más de 20 o 30 minutos sin saber si el bus ya había pasado o faltaba bastante tiempo para que llegara, esa falta de información no solo me fastidiaba sino que me hacía perder tiempo valioso y, literalmente, se degradaba mi día a día. Un día perdí mi segundo bus consecutivo porque llegaron 15 o 20 minutos antes de lo habitual y nadie lo notó, en ese momento decidí comenzar a plasmar mi frustración en algo sólido. Hable con varios amigos que trabajan en compañías de transporte y me di cuenta que el problema era mucho más grande de lo que imaginaba. En Neiva, por ejemplo, el transporte público moviliza miles de usuarios diarios y sin embargo, la mayoría no tiene acceso a información en tiempo real de los buses. Lo qué realmente me dejó con la boca abierta fue saber que muchas compañías ya utilizaban GPS en motores pero esa data nunca llegaba al usuario final, era como una disposición de algo valioso y no aprovechado. En ese momento pensé: “está tecnología ya existe, el problema está claro… ¿por qué no hacer algo para unir estos dos mundos?”

Una de las partes más difíciles, pero, irónicamente, más interesantes, fue el diseño de la aplicación del conductor. Al principio, imaginaba una herramienta complicada, con varias pantallas, múltiples ajustes y un panel lleno de métricas. Pensaba que los conductores querrían ver estadísticas detalladas, informes completos y herramientas de control precisas. Sin embargo, después de entrevistar a algunos conductores de TransMilenio en las primeras semanas, entendí que me estaba enfocando mal. A Carlos, que conducía TransMilenio durante 15 años, no le gustaba realmente mi primer diseño: “Necesitamos algo estúpidamente sencillo. No puedo mirar esta pantalla mientras estoy conduciendo “. Otra conductora, María, me dijo algo más significativo: “Mi trabajo es conducir, no configurar una aplicación”. Esto me abrió los ojos. Me di cuenta de que estaba diseñando desde el punto de vista del desarrollador, no del usuario. El conductor de TransMilenio está en un entorno de alta concentración, donde cualquier distractividad puede ser peligrosa no solo para él sino también para los pasajeros y otros autos en la carretera. Decidimos hacer una aplicación minimalista basada en fricción cero: el conductor abre la aplicación, elige su ruta de una lista y no hace nada más. El GPS está en el fondo. No hay botones adicionales ni menús largos, no hay flujos que requieran más de dos toques. La aplicación se esconde automáticamente y vibra o emite un pequeño pitido si necesita confirmación del conductor.

Pero la recompensa real del proyecto se reveló cuando probamos la aplicación con el usuario final. Durante las pruebas de amigos y miembros de la familia, hubo un momento en que me dije: “Es genial. Ahora estamos empezando a lo grande.” Mi hermana, que siempre detestaba esperar interminables minutos en la línea de la estación El Tiempo, fue una de las primeras en usar la aplicación. Solo una semana después, me envió un mensaje: “¡Esto es increíble! Ahora sé cuándo llega el bus. ¡No me he perdido una vez! ”. Lo mismo sucedió cuando personas como mi madre, una mujer de 52 años sin formación técnica, quisieron probarla. Resultó ser bastante simple para ella. Incluso mi vecino, profesor en la universidad local, me dijo: “Esto debería estar en todos los sistemas de transporte del país”. En el caso de los administradores, era fundamental crear un panel de control eficiente que no los abrumara con datos. Según los supervisores de rutas, los paneles habituales generalmente leen más notificaciones que un oficial de servicio de incendios. Por eso elegimos mostrar solo lo esencial: estado en vivo de cada bus, retrasos, incidentes y un conjunto de opciones rápidas para descubrir y resolver problemas en dos o tres clics.

La parte técnica fue un verdadero problema, y me despertó un par de noches. Era esencial elegir las tecnologías correctas; si la base de algo era inestable, entonces todos los demás correrían el riesgo. Java y Spring Boot se iniciaron en la parte trasera; era el enfoque que mejor conocía, y sabía que necesitaba algo de hierro para manejar los datos casi en tiempo real. Sin embargo, surgió una pregunta práctica: ¿cómo puede una aplicación asegurarse de que los buses siempre puedan comunicarse con el servidor, independientemente de la calidad de la conexión? La calidad de la conexión fue la pesadilla principal: ciertas áreas de Bogotá cuentan con interiores de carreteras amplios, por ejemplo, los túneles o los intercambiadores. Durante una de las pruebas, descubrí que realmente perdía la señal del GPS en el intercambio de la 68 con la 100. Me pregunté si era realista esperar no perder ese tiempo. Después de algunas semanas, encontré MQTT: un protocolo de mensajes ¿Ideal para dispositivos móviles, incluso si mantiene las conexiones activas y funciona sin pérdida de mensajes si se desconecta durante un corto tiempo? Un error perfecto. Al principio, todo ese broker, temas, QoS parecía un dolor, pero una vez que lo entendí. Si un bus pasa por una zona donde no hay señal, entonces todo el sistema sincroniza toda la información, sin parar por una persona.. La protección también se convirtió en un pilar sólido. No creía que, dado que se trataba de un triciclo; se use transporte público, sería necesario algún nivel de aseguramiento. Pero luego amigos con experiencia en seguridad informática me explicaron la tentación que tendrían hackers ilegales para manipular los datos de género manipulando los datos de ubicación. Autenticación con JWT; Regulaciones estrictas y registro de eventos. Para los operadores, introduje la doble autorización: confirma la identidad con un código SMS.

Otro enfoque clave fue la confiabilidad de los datos. De nada sirve que muestres información en vivo si no es precisa. Así que agregamos verificaciones automáticas de coordenadas, detección de ubicaciones imposibles y sistemas de suavizado que rechazan lecturas incorrectas. Cada vez que se envíe una posición, se validará desde mapas y rutas autorizadas antes de mostrarla al usuario. El día de la primera pueba real estaba lleno de preguntas. Después de tantos meses era el momento de probar si todo estaba funcionando. Realicé las primeras pruebas con cinco teléfonos viejos que tenía guardados, simulando rutas por Bogotá mientras caminaba por distintos barrios con la computadora ubicada en una bicicleta. Cada actualización correcta de la pantalla del computador era un triunfo. Los resultados fueron mucho mejores de lo que esperábamos: latencias menores a dos segundos, reconexiones automáticas impecables y funcionamiento estable desde el primer segundo, incluso cuando abría la puerta del ascensor que generaba condiciones de señal pobre. Luego, hicimos pruebas con usuarios reales: familiares y amigos durante una semana. La precisión superó el 95%. El tiempo de espera en paraderos disminuyó entre 20% y 40%. Pero además, lo que más me llenó, fue cómo los usuarios comenzaron a tomar mejores decisiones por ellas mismas. “Ahora sí sé si puedo comprar un café antes de que pase el bus”, me dijo Alejandro, uno de los primeros testers. También tuvimos muy buenos resultados al utilizarlo como administradores. Simulamos desvíos de ruta, y el sistema detectó todos en menos de 30 segundos, y envió inmediatamente alertas. Al final, UrbanTracker nació de un problema diario; mi frustración por los buses perdidos. Pero resultó siendo algo mucho más grande. Es un ejemplo de cómo la mejor tecnología a menudo se produce en situaciones tan simples que enfrentamos día tras día.

Lo que me anima en ese sentido es lo que ya he podido demostrar: la innovación no debería reservarse a empresas punteras. Si tienes las herramientas adecuadas y las usas creativamente para resolver problemas reales, cualquiera puede hacer algo que importe. UrbanTracker ya es una plataforma plenamente funcional: los usuarios pueden entrar en el navegador, los conductores en su app en Android y los administradores en un panel completo de control. Y eso no es sino el principio. Nuestro sistema fue desarrollado desde cero para escalar: prevé la llegada de viajes, rutas optimizadas para el tráfico, integración con otra forma de transporte y mucho más. Si la gente confía más en el transporte público debido a información precisa y accesible, podemos reducir la congestión, la contaminación y mejorar la calidad de vida en nuestras ciudades. UrbanTracker encarna una forma más humana y práctica de ver la tecnología – no como un juguete es un sueño lejano para la gente corriente, sino como una forma de superar nuestras verdaderas dificultades. Quiero que más desarrolladores, más estudiantes y más entusiastas miren a su alrededor, identifiquen los problemas y se animen a resolverlos.
