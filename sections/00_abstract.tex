UrbanTracker es una plataforma integral creada con el propósito de mejorar la movilidad urbana mediante el seguimiento en tiempo real de los vehículos que conforman el sistema de transporte público. La idea surge de la necesidad evidente de ofrecer información más precisa y actualizada sobre el paradero de los autobuses, ya que en la mayoría de ciudades los usuarios siguen dependiendo de estimaciones aproximadas o de suposiciones basadas en la experiencia. Al mostrar la ubicación exacta de cada vehículo, el sistema reduce considerablemente la incertidumbre de los tiempos de espera y permite que los pasajeros puedan planear mejor sus desplazamientos diarios. Paralelamente, la plataforma brinda a los administradores un mayor control operativo, puesto que pueden monitorear el comportamiento de la flota, detectar retrasos con rapidez y tomar decisiones que mejoren la eficacia del servicio.

El funcionamiento de UrbanTracker se basa en varios componentes que trabajan de forma coordinada. Uno de ellos es la aplicación móvil destinada a los conductores, desarrollada en React Native, que tiene como tarea principal obtener las coordenadas GPS del dispositivo y enviarlas de manera continua al servidor central. Esta aplicación se diseñó para que su uso fuera lo más sencillo posible: el conductor únicamente debe iniciar sesión, seleccionar la ruta asignada y continuar con su labor habitual, ya que el sistema se encarga automáticamente de transmitir la información necesaria sin distraerlo de la conducción.

En cuanto a la experiencia del usuario final y de los encargados de la operación, la plataforma web de UrbanTracker permite visualizar en un mapa interactivo las rutas disponibles y los vehículos que se encuentran actualmente activos. Este recurso facilita que las personas identifiquen rápidamente la posición de los autobuses y organicen sus trayectos con datos reales y no con aproximaciones. Para los administradores, la herramienta incluye opciones adicionales como crear o editar rutas, registrar nuevos vehículos, gestionar conductores y observar el estado operativo de cada unidad en tiempo real, todo en una misma interfaz.

Detrás de estas funcionalidades se encuentra un backend desarrollado en Java con el framework Spring Boot, donde reside la lógica de negocio que articula todos los módulos del sistema. La comunicación entre la aplicación móvil, el servidor y la plataforma web se realiza utilizando MQTT, un protocolo ampliamente empleado en soluciones de Internet de las Cosas debido a su capacidad para manejar transmisiones constantes de datos sin requerir grandes recursos de red. Esta elección tecnológica resulta especialmente útil en entornos de transporte urbano, donde la conectividad de los dispositivos móviles puede variar con frecuencia. Al utilizar MQTT, UrbanTracker logra mantener un flujo estable de información incluso en condiciones de red poco favorables.

Además de mostrar ubicaciones en tiempo real, el sistema incorpora mecanismos de autenticación, validación de rutas, almacenamiento estructurado y procesamiento continuo de información para garantizar que los datos sean confiables y estén siempre actualizados. A lo largo del trabajo se explican los requisitos que guiaron el diseño de cada componente, el marco conceptual relacionado con la geolocalización y los protocolos de comunicación utilizados, así como la metodología que se siguió durante la implementación.

Los resultados preliminares obtenidos en las pruebas muestran que la arquitectura elegida —basada en React Native, Spring Boot y MQTT— permite alcanzar tiempos de respuesta muy bajos, lo que se traduce en actualizaciones rápidas y estables. En la práctica, esto mejora tanto la experiencia de quienes usan el sistema para conocer la ubicación de los autobuses como la gestión que realizan los administradores del servicio. En conjunto, estos resultados confirman que UrbanTracker es una solución viable y efectiva para fortalecer la calidad del transporte público mediante tecnologías modernas y accesibles.